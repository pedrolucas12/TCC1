\begin{resumo}
A dengue representa um dos maiores desafios de saúde pública no Brasil, com epidemias cíclicas que sobrecarregam o sistema de saúde. Em 2024, o Distrito Federal enfrentou um surto sem precedentes. Este trabalho investiga a correlação entre variáveis climáticas (precipitação, temperatura e umidade) e a incidência de dengue no DF durante este ano epidêmico. Utilizando dados do SINAN e do INMET, foi realizada uma análise exploratória que revelou uma correlação positiva significativa ($r \approx 0.51$) entre a umidade relativa do ar e o número de casos, superando a influência da precipitação. Os resultados preliminares indicam que a modelagem preditiva baseada em Inteligência Artificial (XGBoost, LSTM), proposta para a próxima etapa deste estudo, é viável e promissora para antecipar surtos locais.

 \vspace{\onelineskip}
 
 \noindent
 \textbf{Palavras-chave}: Dengue. Distrito Federal. Inteligência Artificial. Clima. Epidemiologia.
\end{resumo}
