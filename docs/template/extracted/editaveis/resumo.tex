\begin{resumo}
    A dengue representa um dos desafios de saúde pública no Brasil, com epidemias cíclicas que sobrecarregam o sistema de saúde. Em 2024, o Distrito Federal enfrentou um surto sem precedentes históricos recentes. Este trabalho propõe uma metodologia para o desenvolvimento de modelos preditivos de surtos de dengue utilizando séries temporais e dados climáticos. A metodologia proposta compreende quatro fases sequenciais: (1) Consolidação e expansão da base de dados, incluindo coleta e processamento de dados meteorológicos do INMET (período de 2000 a 2025) e dados epidemiológicos do SINAN via InfoDengue (desde 2007), com integração e unificação dos datasets; (2) Engenharia de atributos e análise exploratória, com construção de atributos defasados (lags de 1 a 12 semanas), atributos derivados (médias móveis, anomalias climáticas, indicadores sazonais) e análise estatística exploratória incluindo correlações e testes de causalidade de Granger; (3) Desenvolvimento de modelos preditivos, abrangendo modelos estatísticos (SARIMA, SARIMAX, Prophet), modelos de aprendizado de máquina (Random Forest, XGBoost) e modelos de deep learning (LSTM, BiLSTM, CNN-LSTM), com seleção automática de hiperparâmetros e validação temporal walk-forward; (4) Integração em sistema de alerta e deploy, incluindo desenvolvimento de API REST, dashboard interativo e automação de atualização semanal. A metodologia visa fornecer uma ferramenta de alerta antecipado quantitativamente validada para previsão semanal de casos de dengue, inicialmente para o Distrito Federal e posteriormente expandida para todos os municípios brasileiros com dados disponíveis.
    
     \vspace{\onelineskip}
     
     \noindent
     \textbf{Palavras-chave}: Dengue. Distrito Federal. Inteligência Artificial. Clima. Epidemiologia.
    \end{resumo}
