\chapter[Conclusões e Próximos Passos]{Conclusões e Próximos Passos}
\label{cap:conclusao}

Este capítulo sintetiza as contribuições alcançadas no TCC 1, destaca as limitações encontradas e apresenta o plano de continuidade para o TCC 2.

\section{Conclusões do TCC 1}

\begin{itemize}
    \item \textbf{Diagnóstico do problema}: foi elaborada uma visão abrangente do cenário epidemiológico da dengue no Brasil, com respaldo na literatura atualizada e análise dos microdados de 2025.
    \item \textbf{Pipeline de dados}: implementou-se uma base reprodutível para coleta e processamento do SINAN, bem como scripts preparatórios para integrações climáticas (INMET, CHIRPS, ERA5, NASA POWER).
    \item \textbf{Análise exploratória}: geraram-se estatísticas e visualizações que revelam padrões sazonais, desigualdades territoriais e indicadores clínicos relevantes (hospitalizações, óbitos, sintomas).
    \item \textbf{Planejamento metodológico}: definiu-se um pipeline completo de modelagem, métricas e técnicas de interpretação que balizarão a fase de experimentos.
    \item \textbf{Estruturação do TCC}: o texto foi esboçado no template oficial, facilitando a evolução para uma versão pronta para a banca.
\end{itemize}

\section{Limitações Identificadas}

\begin{itemize}
    \item Dados climáticos ainda estão em fase de consolidação; será necessário executar o processamento completo do INMET e incorporar CHIRPS/ERA5.
    \item Parte das rotinas de visualização precisa ser automatizada no notebook principal para facilitar reprodutibilidade.
    \item A etapa de modelagem comparativa ainda não foi iniciada; foram definidos apenas os parâmetros iniciais.
    \item Informações complementares (nomes completos dos autores, orientador, banca, CDU) devem ser confirmadas com a coordenação.
\end{itemize}

\section{Próximos Passos para o TCC 2}

\begin{enumerate}
    \item Finalizar a consolidação de dados climáticos, incluindo interpolação espacial e tratamento de valores faltantes.
    \item Implementar os modelos baseline, de aprendizado de máquina e de aprendizado profundo com validação temporal.
    \item Avaliar o desempenho com múltiplas métricas e realizar interpretação com SHAP e análise de lags.
    \item Desenvolver o protótipo do sistema de alerta e integrar visualizações interativas (dashboard ou API).
    \item Refinar o texto do TCC com os resultados completos, incluir anexos técnicos (código, tabelas) e preparar a versão para defesa.
\end{enumerate}

A continuidade do trabalho enfocará a execução rigorosa dos experimentos, a consolidação de evidências quantitativas e a entrega de uma ferramenta operável para gestores de saúde.

