\begin{resumo}[Abstract]
Dengue fever represents one of the major public health challenges in Brazil, with cyclical epidemics burdening the healthcare system. In 2024, the Federal District (DF) faced an unprecedented outbreak. This work investigates the correlation between climatic variables (precipitation, temperature, and humidity) and dengue incidence in the DF during this epidemic year. Using data from SINAN and INMET, an exploratory analysis revealed a significant positive correlation ($r \approx 0.51$) between relative humidity and the number of cases, surpassing the influence of precipitation. Preliminary results indicate that predictive modeling based on Artificial Intelligence (XGBoost, LSTM), proposed for the next stage of this study, is feasible and promising for anticipating local outbreaks.

 \vspace{\onelineskip}
 
 \noindent
 \textbf{Keywords}: Dengue. Federal District. Artificial Intelligence. Climate. Epidemiology.
\end{resumo}
