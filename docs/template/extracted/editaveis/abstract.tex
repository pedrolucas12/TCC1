\begin{resumo}[Abstract]
    Dengue fever represents one of the major public health challenges in Brazil, with cyclical epidemics burdening the healthcare system. In 2024, the Federal District (DF) faced an unprecedented outbreak in recent history. This work proposes a methodology for developing predictive models of dengue outbreaks using time series and climatic data. The proposed methodology comprises four sequential phases: (1) Data consolidation and expansion, including collection and processing of meteorological data from INMET (period 2000 to 2025) and epidemiological data from SINAN via InfoDengue (since 2007), with integration and unification of datasets; (2) Feature engineering and exploratory analysis, with construction of lagged features (lags of 1 to 12 weeks), derived attributes (moving averages, climatic anomalies, seasonal indicators) and exploratory statistical analysis including correlations and Granger causality tests; (3) Development of predictive models, encompassing statistical models (SARIMA, SARIMAX, Prophet), machine learning models (Random Forest, XGBoost) and deep learning models (LSTM, BiLSTM, CNN-LSTM), with automatic hyperparameter selection and walk-forward temporal validation; (4) Integration into alert system and deployment, including REST API development, interactive dashboard and weekly update automation. The methodology aims to provide a quantitatively validated early warning tool for weekly dengue case forecasting, initially for the Federal District and subsequently expanded to all Brazilian municipalities with available data.
    
     \vspace{\onelineskip}
     
     \noindent
     \textbf{Keywords}: Dengue. Federal District. Artificial Intelligence. Climate. Epidemiology.
    \end{resumo}
