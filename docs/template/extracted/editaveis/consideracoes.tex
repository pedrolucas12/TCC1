\chapter{Considerações Finais e Próximos Passos}
\label{cap:consideracoes}

\section{Conclusão Parcial}

O presente trabalho atingiu seu objetivo inicial de construir uma base de dados sólida e unificada, integrando informações epidemiológicas e climáticas do Distrito Federal para o período crítico de 2022 a 2024.

As análises realizadas permitiram concluir que:
\begin{enumerate}
    \item O surto de 2024 foi um evento estatístico extremo (\textit{outlier}), com uma magnitude 3 a 4 vezes superior aos anos anteriores, evidenciando a urgência de modelos preditivos mais robustos.
    \item As variáveis climáticas não agem de forma imediata. A análise de \textit{lags} confirmou que a precipitação possui um "efeito gatilho" que se manifesta com maior intensidade nos casos de dengue cerca de 7 a 8 semanas depois.
    \item A umidade relativa do ar mostrou-se o indicador ambiental mais correlacionado com a doença no DF, refletindo as características do bioma Cerrado.
\end{enumerate}

Esses achados validam a hipótese de pesquisa de que a integração de dados climáticos defasados é essencial para explicar a dinâmica da doença.

\section{Plano de Trabalho para o TCC 2}

Com a base de dados consolidada e as variáveis explicativas (features) selecionadas, o TCC 2 focará exclusivamente na implementação e otimização dos modelos preditivos. As etapas previstas são:

\begin{itemize}
    \item \textbf{Engenharia de Atributos (Feature Engineering):} Criação explícita das variáveis de \textit{lag} (ex: `rain_lag_4`, `temp_lag_4`) no dataset de treinamento.
    \item \textbf{Treinamento dos Modelos:}
    \begin{itemize}
        \item \textbf{SARIMA:} Ajuste fino dos parâmetros $(p,d,q)(P,D,Q)_s$ para capturar a sazonalidade.
        \item \textbf{XGBoost:} Treinamento com foco em capturar as não-linearidades e interações entre variáveis climáticas.
        \item \textbf{LSTM (Redes Neurais Recorrentes):} Implementação de arquitetura para aprendizado de sequências temporais longas.
    \end{itemize}
    \item \textbf{Avaliação Rigorosa:} Utilização de validação cruzada em janelas temporais (\textit{Time Series Cross-Validation}) para evitar vazamento de dados do futuro para o passado.
    \item \textbf{Comparativo:} Definição do melhor modelo com base nas métricas RMSE (Raiz do Erro Quadrático Médio) e MAE (Erro Absoluto Médio).
\end{itemize}

Espera-se que o produto final seja um modelo capaz de emitir alertas antecipados com pelo menos 4 semanas de antecedência, auxiliando os gestores de saúde na tomada de decisão.
