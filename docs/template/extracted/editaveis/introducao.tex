\chapter[Introdução]{Introdução}
\label{cap:introducao}

\section{Importância do Trabalho e Contextualização}

\subsection{Por que este trabalho é importante?}

A dengue representa um dos maiores desafios de saúde pública global da atualidade, especialmente em regiões tropicais e subtropicais. A Organização Mundial da Saúde (OMS) estima que aproximadamente 3,9 bilhões de pessoas — equivalente a metade da população mundial — vivem em áreas com risco de infecção por dengue. A incidência global da doença aumentou vertiginosamente nas últimas quatro décadas, passando de cerca de meio milhão de casos notificados em 2000 para mais de 5,2 milhões em 2019, representando um crescimento superior a 1000\%.

No Brasil, o cenário epidemiológico é particularmente alarmante e caracteriza-se por uma hiperendemicidade, com a co-circulação simultânea dos quatro sorotipos virais (DENV-1, DENV-2, DENV-3 e DENV-4). O país enfrenta epidemias cíclicas e sazonais que impõem uma carga severa e recorrente ao Sistema Único de Saúde (SUS). Dados oficiais do Ministério da Saúde revelam que, apenas no primeiro semestre de 2024, foram notificados mais de 6 milhões de casos prováveis de dengue em todo o território nacional, superando os registros históricos anteriores e evidenciando a dimensão continental do problema sanitário.

O Distrito Federal (DF), objeto de estudo desta pesquisa, tornou-se um cenário emblemático e crítico da gravidade desta crise sanitária nos últimos anos. Localizado no bioma Cerrado, o DF apresenta características climáticas específicas que modulam sazonalmente a dinâmica de transmissão: um regime pluviométrico bem definido, alternando entre uma estação seca (aproximadamente de maio a setembro) e uma chuvosa (outubro a abril), criando condições cíclicas ideais para a proliferação do vetor \textit{Aedes aegypti}.

Dados consolidados e processados para esta pesquisa, obtidos através da integração sistemática das bases do Sistema de Informação de Agravos de Notificação (SINAN) e do Instituto Nacional de Meteorologia (INMET), revelam que, no triênio compreendido entre 2022 e 2024, o Distrito Federal acumulou um total de \textbf{463.560 casos notificados} de dengue. 

A análise temporal desses dados revela uma disparidade impressionante entre os anos estudados. Enquanto 2022 e 2023 apresentaram perfis epidemiológicos relativamente controlados — com 85.346 e 64.897 casos, respectivamente —, o ano de 2024 caracterizou-se por uma epidemia explosiva e sem precedentes históricos, totalizando \textbf{313.317 casos} notificados. Esse volume representa um aumento superior a 360\% em relação ao ano anterior (64.897 casos em 2023) e supera amplamente a soma de toda a série histórica recente dos anos anteriores. A média semanal de casos saltou de aproximadamente 1.247 casos por semana em 2022 para 5.927 casos por semana em 2024, evidenciando uma mudança de escala epidemiológica dramática.

O pico dessa crise ocorreu na semana epidemiológica iniciada em 18 de fevereiro de 2024, quando o sistema de saúde registrou mais de \textbf{25.714 casos em apenas sete dias} — uma magnitude que evidencia não apenas a incapacidade dos métodos tradicionais de vigilância epidemiológica em antecipar eventos extremos, mas também o colapso subsequente e imediato dos serviços de saúde. Relatos da mídia e documentos oficiais da Secretaria de Saúde do DF descreveram unidades hospitalares operando além da capacidade, com pacientes aguardando atendimento em corredores, esgotamento de insumos básicos (soro fisiológico, medicamentos para controle de dor e febre), e comprometimento do atendimento a outras patologias urgentes, incluindo emergências cardíacas e neurológicas.

Este trabalho é importante porque busca transicionar o paradigma da vigilância epidemiológica de uma abordagem \textbf{reativa} ("o que aconteceu?") para uma abordagem \textbf{preditiva e proativa} ("o que vai acontecer e quando?"). Os métodos tradicionais de vigilância operam com base em notificações clínicas e laboratoriais, que são naturalmente defasadas em relação ao momento real da transmissão viral na comunidade. Existe um intervalo crítico de várias semanas entre a ocorrência de um evento ambiental (ex: chuva intensa) e a manifestação clínica dos primeiros casos, período durante o qual a transmissão já está ocorrendo silenciosamente.

Modelos preditivos baseados em Inteligência Artificial, capazes de antecipar surtos com 4 a 8 semanas de antecedência, oferecem aos gestores públicos uma janela de oportunidade valiosa para:
\begin{itemize}
    \item \textbf{Otimizar a logística de insumos:} Estocar adequadamente medicamentos, testes diagnósticos e equipamentos hospitalares antes da saturação da demanda, evitando desabastecimento crítico.
    \item \textbf{Dimensionar equipes de saúde:} Contratar temporariamente profissionais de saúde, redistribuir equipes das unidades básicas para áreas de maior risco, e organizar plantões extras nas unidades de pronto-atendimento.
    \item \textbf{Direcionar ações de controle vetorial:} Priorizar bloqueios químicos (aplicação de inseticidas) e visitas domiciliares de agentes de endemias para áreas identificadas como de risco iminente antes da explosão de casos, maximizando o impacto das intervenções quando ainda há tempo para prevenir novas infecções.
    \item \textbf{Implementar estratégias de comunicação pública:} Lançar campanhas educativas e de mobilização social em tempo hábil, alertando a população sobre medidas preventivas e sinais de alarme, potencialmente reduzindo a procura desnecessária por serviços de saúde.
\end{itemize}

A implementação bem-sucedida de sistemas de alerta antecipado pode não apenas salvar vidas, mas também reduzir substancialmente o impacto econômico da doença, que inclui custos diretos (tratamento, hospitalização) e indiretos (perda de produtividade, absenteísmo escolar e laboral), estimados em bilhões de reais anuais no Brasil.

\subsection{Do que se trata o assunto abordado?}

A dengue é uma arbovirose sistêmica de evolução benigna na maioria dos casos (aproximadamente 70-80\%), mas que pode apresentar manifestações graves e potencialmente letais. O agente etiológico é um vírus de RNA de fita simples e polaridade positiva (DENV), pertencente à família \textit{Flaviviridae} e ao gênero \textit{Flavivirus}, que compartilha características filogenéticas com outros vírus de importância médica, como os causadores da febre amarela, Zika e chikungunya.

O vírus da dengue possui quatro sorotipos antigenicamente distintos e geneticamente relacionados (DENV-1, DENV-2, DENV-3 e DENV-4), cada um com características epidemiológicas e patogênicas próprias. Estudos moleculares revelam que esses sorotipos divergiram de um ancestral comum há aproximadamente 1.000 anos, e sua diferenciação continua através de mutações acumuladas ao longo do tempo, gerando diferentes genótipos e linhagens dentro de cada sorotipo.

A transmissão ocorre primariamente pela picada de fêmeas infectadas de mosquitos do gênero \textit{Aedes}, sendo o \textit{Aedes aegypti} o vetor primário em áreas urbanas e o \textit{Aedes albopictus} um vetor secundário com potencial crescente de adaptação a ambientes periurbanos e rurais. O \textit{Aedes aegypti} é uma espécie altamente antropofílica, demonstrando preferência marcada por se alimentar de sangue humano, o que aumenta sua eficiência como vetor em áreas urbanas densamente povoadas. O mosquito se infecta ao picar um indivíduo virêmico (portador do vírus na corrente sanguínea) e, após um período de incubação extrínseco (PIE) que varia de 8 a 12 dias dependendo da temperatura ambiente, torna-se capaz de transmitir o vírus a novos hospedeiros humanos durante múltiplos repastos sanguíneos ao longo de sua vida adulta, que pode durar de 2 a 4 semanas em condições ideais.

A infecção por um sorotipo confere imunidade permanente e específica apenas para aquele sorotipo particular, havendo a possibilidade de infecções subsequentes pelos demais sorotipos. Essa característica imunológica aumenta significativamente o risco de desenvolvimento de formas graves da doença em infecções secundárias. A teoria mais aceita, conhecida como "hipótese de potência dependente de anticorpos" (Antibody-Dependent Enhancement - ADE), postula que anticorpos gerados em uma primeira infecção podem, paradoxalmente, facilitar a entrada do vírus heterotípico nas células do sistema imune (monócitos e macrófagos) durante uma segunda infecção, aumentando a carga viral e desencadeando uma resposta imune exacerbada. Essa resposta imune hiperativa pode levar a complicações graves, como extravasamento plasmático maciço, hemorragias espontâneas e choque hipovolêmico, características da Dengue Grave (anteriormente denominada Dengue Hemorrágica), que pode evoluir para a síndrome do choque da dengue e óbito caso não seja tratada adequadamente e em tempo hábil.

\subsubsection{Gravidade da doença e limitações impostas aos pacientes}

Os pacientes com dengue enfrentam limitações físicas, funcionais e sociais significativas durante o período da doença. A fase febril inicial, que dura tipicamente de 3 a 7 dias, caracteriza-se por sintomas súbitos e intensos que aparecem abruptamente após o período de incubação intrínseca (4 a 10 dias após a picada do mosquito infectado). A febre alta súbita (38-40°C), muitas vezes bifásica (com dois picos de temperatura), é acompanhada de dores musculares intensas e generalizadas que os pacientes frequentemente descrevem como "dor nos ossos", cefaleia severa e persistente, e dor retroorbital (atrás dos olhos) que piora com movimentos oculares. Esses sintomas comprometem drasticamente a capacidade de realizar atividades cotidianas básicas, como trabalhar, estudar ou mesmo realizar tarefas domésticas simples.

Em casos graves, que ocorrem em aproximadamente 5-10\% das infecções e são mais comuns em infecções secundárias, os pacientes podem apresentar manifestações clínicas críticas:

\begin{itemize}
    \item \textbf{Extravasamento de plasma:} O mecanismo fisiopatológico central da dengue grave é o aumento da permeabilidade vascular, resultando em extravasamento de plasma para os espaços extravasculares. Clinicamente, isso se manifesta como derrame pleural (acúmulo de líquido na cavidade torácica), ascite (acúmulo de líquido no abdome), derrame pericárdico e edema generalizado. Em casos severos, a perda de volume plasmático pode ultrapassar 20\% do volume sanguíneo total, levando a hipotensão arterial, taquicardia compensatória e, em última instância, choque hipovolêmico, exigindo hospitalização imediata em unidade de terapia intensiva, reposição volêmica agressiva com soluções cristaloides ou coloides, e monitoramento hemodinâmico contínuo.
    
    \item \textbf{Hemorragias:} A trombocitopenia (diminuição do número de plaquetas) e as alterações na função plaquetária, combinadas com distúrbios da coagulação, podem resultar em sangramentos espontâneos ou após procedimentos médicos (venopunção, injeções intramusculares). As manifestações hemorrágicas variam desde petéquias cutâneas e epistaxe (sangramento nasal) até hemorragia gastrointestinal, sangramento uterino anormal e, em casos extremos, hemorragia intracraniana, que pode ser fatal se não diagnosticada e tratada prontamente. Pacientes com sangramentos ativos podem necessitar de transfusão de plaquetas e plasma fresco congelado.
    
    \item \textbf{Comprometimento de órgãos:} Embora menos comum, a dengue pode causar disfunção orgânica múltipla. Hepatite por dengue, caracterizada por elevação de transaminases e icterícia, pode levar a insuficiência hepática aguda. A encefalite por dengue, resultante da invasão viral do sistema nervoso central ou de complicações da síndrome de choque, pode causar convulsões, alteração do nível de consciência e déficits neurológicos permanentes. Miocardite e insuficiência cardíaca aguda têm sido relatadas, especialmente em crianças. Insuficiência renal aguda pode ocorrer secundária ao choque ou diretamente por nefrotoxicidade viral, exigindo terapia dialítica temporária em casos severos.
\end{itemize}

Além das manifestações clínicas agudas, muitos pacientes desenvolvem sequelas de longo prazo que podem persistir por semanas ou meses após a resolução da fase aguda da infecção. A síndrome pós-dengue, ainda pouco estudada mas frequentemente relatada por pacientes e profissionais de saúde, caracteriza-se por fadiga crônica incapacitante, dores articulares e musculares persistentes, dificuldades de concentração e memória (popularmente conhecida como "neblina mental"), e depressão secundária. Estudos longitudinais sugerem que até 40\% dos pacientes que tiveram dengue grave podem apresentar sintomas persistentes após 3 meses da infecção aguda, impactando significativamente a qualidade de vida e a capacidade de retorno às atividades profissionais e acadêmicas.

\subsubsection{Perigos e impacto social}

O impacto da dengue transcende drasticamente a esfera clínica individual, gerando um fardo multidimensional e pesado para a sociedade, a economia e o sistema de saúde. Epidemias de grande magnitude, como a observada no Distrito Federal em 2024, provocam uma cascata de efeitos deletérios que se propagam através de múltiplas camadas sociais e econômicas.

A sobrecarga do Sistema Único de Saúde (SUS) é talvez o impacto mais visível e imediato. Durante picos epidêmicos, as unidades de atenção primária, pronto-atendimento e emergências hospitalares são inundadas com pacientes procurando atendimento, muitas vezes com sintomas leves ou moderados que poderiam ser manejados ambulatorialmente, mas que geram ansiedade e procura por serviços de saúde devido ao medo de complicações. Isso resulta em tempos de espera que podem ultrapassar 8-12 horas, comprometendo a qualidade do atendimento e aumentando o risco de erros médicos por sobrecarga dos profissionais. Leitos hospitalares, especialmente em unidades de terapia intensiva pediátrica e adulta, são rapidamente ocupados por pacientes com dengue grave, reduzindo a disponibilidade para outras emergências médicas, incluindo acidentes vasculares cerebrais, infartos do miocárdio e traumas.

O absenteísmo escolar e laboral durante epidemias atinge níveis críticos. Estudos econômicos estimam que, durante a epidemia de 2024 no DF, milhares de dias letivos foram perdidos por estudantes que estiveram doentes ou cuidando de familiares doentes. No ambiente de trabalho, a perda de produtividade é ainda mais significativa: funcionários doentes ficam afastados por períodos que variam de 5 a 15 dias, dependendo da gravidade, enquanto outros podem precisar se ausentar para cuidar de filhos ou parentes doentes. Em setores essenciais, como educação, segurança pública e saúde, essas ausências podem comprometer o funcionamento básico dos serviços.

Os custos econômicos diretos e indiretos são astronômicos. Estudos de análise de custo-efetividade estimam que o custo médio por caso de dengue no Brasil varia de R\$ 500 a R\$ 2.000, dependendo da complexidade do caso e da necessidade de hospitalização. Multiplicando esses valores pelos milhões de casos anuais, chega-se a um impacto econômico que pode ultrapassar R\$ 5 bilhões anualmente. Os custos diretos incluem gastos com consultas médicas, exames laboratoriais, medicamentos, internações hospitalares e procedimentos diagnósticos e terapêuticos. Os custos indiretos, frequentemente subestimados, incluem perda de produtividade da força de trabalho, custos previdenciários por afastamentos, impacto no turismo (áreas afetadas podem ter redução de visitantes), e custos de oportunidade (recursos que poderiam ser investidos em outras áreas da saúde pública).

Populações vulneráveis, especialmente aquelas em áreas urbanas periféricas com infraestrutura sanitária precária, são desproporcionalmente afetadas, exacerbando desigualdades sociais já existentes. Nessas comunidades, fatores como falta de abastecimento regular de água (levando ao armazenamento doméstico em recipientes que se tornam criadouros), coleta irregular de lixo (que acumula materiais que retêm água da chuva), e habitação precária (com telhados que permitem entrada de água) criam condições ideais para a proliferação do vetor. Além disso, o acesso limitado a serviços de saúde de qualidade, especialmente em horários de crise, significa que casos graves podem não receber atenção adequada em tempo hábil, aumentando o risco de complicações e óbitos evitáveis.

\subsection{Relevância da tecnologia para atacar o problema}

A dinâmica de transmissão da dengue é um fenômeno extraordinariamente complexo, não-linear, estocástico e fortemente modulado por uma multiplicidade de variáveis climáticas, ambientais, sociais e comportamentais que interagem de forma sinérgica e não-aditiva. Compreender e modelar essa complexidade requer ferramentas analíticas que vão além das capacidades dos métodos estatísticos clássicos utilizados tradicionalmente em vigilância epidemiológica.

A temperatura ambiente regula criticamente múltiplos aspectos do ciclo de vida do vetor e do vírus. Em temperaturas ótimas (aproximadamente 28-32°C), o desenvolvimento larvário do \textit{Aedes aegypti} é acelerado, reduzindo o tempo necessário para a emergência de mosquitos adultos de aproximadamente 10-14 dias para 7-9 dias. O Período de Incubação Extrínseco (PIE) do vírus dentro do mosquito — intervalo entre a ingestão do vírus durante um repasto sanguíneo e a capacidade do mosquito de transmitir o vírus em um novo repasto — também é sensivelmente influenciado pela temperatura. Em temperaturas mais baixas (abaixo de 20°C), o PIE pode se estender para 15-20 dias ou mais, enquanto em temperaturas elevadas (acima de 32°C), o PIE pode ser reduzido para 5-7 dias. Essa relação não-linear significa que pequenas variações de temperatura podem ter efeitos exponenciais na taxa de transmissão viral.

A precipitação fornece os criadouros aquáticos essenciais para a oviposição e o desenvolvimento das fases imaturas do mosquito (ovo, larva e pupa). No entanto, a relação entre chuva e densidade vetorial não é linear: chuvas muito intensas podem, paradoxalmente, reduzir temporariamente a população de mosquitos adultos ao eliminar criadouros temporários e perturbar o ambiente, mas chuvas moderadas e frequentes criam condições ideais para a proliferação. Além disso, a distribuição temporal da chuva é crucial: um verão com chuvas concentradas em poucos eventos intensos pode ter efeito diferente de um verão com precipitação distribuída uniformemente.

A umidade relativa do ar influencia diretamente a longevidade do mosquito adulto e sua capacidade de voo e busca por repastos sanguíneos. Em ambientes secos (umidade abaixo de 60\%), os mosquitos desidratam rapidamente e têm sua sobrevivência reduzida, enquanto em ambientes muito úmidos (acima de 85\%), a umidade pode favorecer a sobrevivência e a atividade de voo. A umidade também afeta a taxa de evaporação da água dos criadouros, influenciando a disponibilidade de habitat para as larvas.

Além disso, existe um fenômeno crucial conhecido como "defasagem temporal" ou \textit{lag}: a chuva que ocorre hoje não resulta em mosquitos adultos imediatamente, mas sim semanas depois, após o ciclo completo de eclosão dos ovos, desenvolvimento larvário (4 estágios), fase pupal e emergência de adultos. O tempo total desse ciclo varia de 7 a 14 dias dependendo das condições ambientais. Adicionalmente, após a emergência, o mosquito precisa de alguns dias para atingir maturidade sexual, realizar o primeiro repasto sanguíneo, completar o PIE do vírus (se picou um indivíduo virêmico), e então transmitir o vírus em repastos subsequentes. Considerando também o período de incubação intrínseca no humano (4-10 dias) e o tempo entre o aparecimento de sintomas e a notificação oficial, o intervalo total entre um evento climático (ex: chuva intensa) e o aparecimento de casos notificados pode ser de 4 a 8 semanas.

Os métodos estatísticos clássicos de vigilância epidemiológica, como modelos compartimentais baseados em equações diferenciais (SIR, SEIR) ou modelos univariados de séries temporais (ARIMA), operam de forma reativa e frequentemente falham em capturar essas nuances não-lineares, interações complexas entre múltiplas variáveis exógenas, e dependências temporais de longo prazo. Modelos lineares assumem que as relações entre variáveis são aditivas e proporcionais, o que raramente reflete a realidade biológica. Por exemplo, um modelo linear não consegue capturar que "a chuva só aumenta significativamente a dengue se a temperatura estiver acima de 25°C e a umidade acima de 70\%", uma interação complexa que é biologicamente plausível.

É nesse contexto que a tecnologia de Inteligência Artificial (IA) se torna uma ferramenta não apenas útil, mas indispensável. Algoritmos de Aprendizado de Máquina (\textit{Machine Learning}), como o \textit{Extreme Gradient Boosting} (XGBoost), e de Aprendizado Profundo (\textit{Deep Learning}), como as Redes Neurais Recorrentes LSTM (\textit{Long Short-Term Memory}), possuem capacidades intrínsecas que os tornam ideais para modelar a complexidade da dengue:

\begin{itemize}
    \item \textbf{Captura de não-linearidades complexas:} Diferente de modelos lineares tradicionais, algoritmos como XGBoost constroem estruturas hierárquicas de decisão (árvores) que podem modelar interações de alta ordem entre variáveis. O algoritmo aprende automaticamente que a relação entre chuva e casos de dengue não é simples e direta, mas depende do contexto de outras variáveis, como temperatura, umidade e histórico de casos anteriores.
    
    \item \textbf{Modelagem de dependências temporais de longo prazo:} As arquiteturas LSTM são especificamente projetadas para processar sequências temporais e reter informações relevantes por longos períodos. Isso permite que o modelo "lembre" que um verão muito chuvoso há 3 meses pode estar influenciando os casos atuais, mesmo que as condições climáticas recentes tenham sido diferentes.
    
    \item \textbf{Processamento de múltiplas variáveis simultaneamente:} Algoritmos de IA podem integrar simultaneamente dados de chuva, temperatura média/mínima/máxima, umidade relativa, pressão atmosférica, velocidade do vento, histórico de casos com diferentes defasagens, e até variáveis demográficas ou de mobilidade (se disponíveis), identificando automaticamente quais combinações são mais preditivas através de processos de aprendizado iterativo.
    
    \item \textbf{Antecipação temporal através de engenharia de atributos:} Através da criação sistemática de variáveis defasadas (\textit{lagged features}), os modelos podem aprender que a chuva de 2 meses atrás influencia mais os casos atuais do que a chuva da semana corrente, alinhando-se perfeitamente com a biologia do vetor e o ciclo de transmissão viral.
    
    \item \textbf{Adaptação a padrões emergentes:} Algoritmos de IA podem detectar mudanças sutis em padrões históricos, adaptando-se a novos cenários epidemiológicos, como o surgimento de um novo sorotipo dominante ou mudanças climáticas que alteram a sazonalidade histórica.
\end{itemize}

Essas tecnologias permitem transformar dados brutos climáticos e epidemiológicos — que, individualmente, podem parecer ruidosos ou inconclusivos — em inteligência preditiva acionável, oferecendo aos gestores públicos uma ferramenta quantitativa e objetiva para antecipar surtos e dimensionar recursos proativamente antes que a transmissão atinja seu pico, potencialmente salvando vidas e recursos significativos.

\section{Pergunta de Pesquisa}

Considerando a gravidade do cenário epidemiológico no Distrito Federal, o potencial inexplorado das variáveis ambientais como preditores antecipados, e a necessidade urgente de transicionar de uma vigilância reativa para uma vigilância preditiva baseada em evidências quantitativas, este trabalho propõe-se a responder à seguinte questão central:

\begin{quote}
\textit{Como a integração sistemática de dados climáticos exógenos (precipitação, temperatura média, umidade relativa, pressão atmosférica) com a série temporal histórica de casos de dengue no Distrito Federal, utilizando janelas temporais defasadas (\textit{lags}) otimizadas estatisticamente através de análise de correlação cruzada e testes de causalidade de Granger, pode aprimorar significativamente a capacidade de modelos de Inteligência Artificial (SARIMA, XGBoost e LSTM) em antecipar e dimensionar surtos epidêmicos severos, como o evento extremo observado em 2024, e de que forma essa capacidade preditiva aprimorada pode fundamentar um sistema de alerta antecipado mais eficaz, preciso e robusto que os métodos tradicionais de vigilância epidemiológica baseados exclusivamente em notificações clínicas?}
\end{quote}

Esta pergunta de pesquisa sintetiza de forma abrangente o que está sendo proposto: não apenas desenvolver modelos preditivos como um exercício técnico, mas também investigar cientificamente, através de métodos estatísticos rigorosos, a contribuição real e quantificável das variáveis climáticas defasadas na capacidade de previsão. Além disso, a pergunta busca comparar sistematicamente diferentes arquiteturas de IA (um modelo estatístico clássico, um algoritmo de machine learning baseado em árvores, e uma rede neural profunda) para identificar qual abordagem oferece melhor capacidade de previsão em diferentes cenários, expandindo o conhecimento do campo em modelagem preditiva aplicada à saúde pública e fornecendo evidências científicas sólidas para fundamentar decisões de política pública em vigilância epidemiológica.

\section{Objetivos}

Os objetivos deste trabalho constituem a resposta direta e operacional à pergunta de pesquisa formulada acima, decompondo a questão central em tarefas metodológicas específicas, mensuráveis e executáveis.

\subsection{Objetivo Geral}

Desenvolver, treinar, otimizar e avaliar comparativamente modelos preditivos baseados em Inteligência Artificial (SARIMA, XGBoost e LSTM) para a previsão semanal de casos de dengue no Distrito Federal, utilizando uma base de dados unificada, tratada, validada e enriquecida com variáveis climáticas defasadas temporalmente do período de 2022 a 2024, visando fornecer uma ferramenta de alerta antecipado quantitativamente validada que demonstra capacidade preditiva superior aos métodos tradicionais de vigilância epidemiológica baseados exclusivamente em notificações clínicas tardias, contribuindo para a transição paradigmática da vigilância reativa para a vigilância preditiva e proativa na saúde pública brasileira.

\subsection{Objetivos Específicos}

Para alcançar o objetivo geral de forma estruturada e reprodutível, foram definidos os seguintes objetivos específicos, que representam os passos metodológicos sequenciais e interdependentes a serem seguidos:

\begin{enumerate}
    \item \textbf{Construir um pipeline automatizado, robusto e reprodutível de coleta e integração de dados:} Desenvolver scripts computacionais modulares e bem documentados em linguagem Python para a extração automatizada, limpeza sistemática, validação de qualidade e unificação estrutural de dados epidemiológicos do Sistema de Informação de Agravos de Notificação (SINAN), obtidos via API REST do InfoDengue, e dados meteorológicos históricos do Instituto Nacional de Meteorologia (INMET), referentes especificamente à Estação Meteorológica Automática de Brasília (A001) no período de 2022 a 2024. O pipeline deve garantir a padronização rigorosa de datas (formato ISO 8601), a agregação temporal adequada (conversão de dados diários/horários para granularidade semanal, utilizando soma para precipitação e média para variáveis contínuas), o tratamento de dados faltantes através de métodos apropriados (interpolação linear para variáveis climáticas, identificação e remoção de outliers impossíveis), e a compatibilidade estrutural entre as bases através de merge utilizando a data de início da semana epidemiológica como chave primária.
    
    \item \textbf{Realizar análise exploratória abrangente e análise estatística profunda de correlação e causalidade:} Quantificar sistematicamente a influência das variáveis climáticas (precipitação, temperatura média, umidade relativa, pressão atmosférica) sobre a incidência de dengue através do cálculo de correlações de Pearson (para relações lineares) e Spearman (para relações monotônicas não-lineares), implementar análise de correlação cruzada (Cross-Correlation Function - CCF) para identificar e quantificar os tempos de defasagem temporal (\textit{lags}) mais significativos e biologicamente relevantes para cada variável ambiental, explorando defasagens de 0 a 8 semanas para capturar o ciclo biológico completo do vetor e o período de incubação viral. Adicionalmente, aplicar testes estatísticos formais de causalidade de Granger para validar cientificamente a hipótese de que as variáveis climáticas antecedem temporalmente e melhoram significativamente a capacidade de previsão dos casos de dengue, comparando modelos autoregressivos com e sem a inclusão de valores defasados das covariáveis climáticas, utilizando nível de significância de 5\% e testando defasagens de 1 a 4 semanas. Gerar uma tabela consolidada de correlações e causalidades que sirva como referência para a engenharia de atributos na etapa de modelagem.
    
    \item \textbf{Realizar decomposição temporal avançada e análise detalhada de sazonalidade:} Aplicar decomposição STL (Seasonal and Trend decomposition using Loess) à série temporal de casos de dengue para isolar e analisar separadamente os componentes de tendência de longo prazo, sazonalidade recorrente (ciclos anuais) e resíduo (ruído aleatório e variação não-explicada), permitindo uma compreensão mais profunda e matizada da estrutura temporal dos dados. Identificar padrões sazonais recorrentes (ex: picos consistentes em fevereiro-março), tendências de crescimento ou declínio de longo prazo, e eventos atípicos ou anomalias que se destacam do padrão esperado. Essa análise deve informar diretamente a escolha de hiperparâmetros dos modelos preditivos (ex: ordem de sazonalidade no SARIMA) e validar se os modelos conseguem capturar adequadamente esses componentes estruturais.
    
    \item \textbf{Fundamentar teoricamente e justificar a escolha das arquiteturas de IA selecionadas:} Realizar revisão bibliográfica sistemática e crítica sobre os três algoritmos de previsão de séries temporais selecionados (SARIMA como modelo estatístico clássico de baseline, XGBoost como representante de algoritmos de machine learning baseados em ensemble de árvores de decisão, e LSTM como representante de redes neurais profundas recorrentes), detalhando seus fundamentos matemáticos e computacionais, vantagens específicas para o problema de predição epidemiológica, limitações conhecidas, requisitos de implementação técnica, hiperparâmetros críticos que precisarão ser otimizados, estratégias de validação apropriadas, e evidências de aplicação bem-sucedida em problemas similares na literatura científica. Esta fundamentação teórica deve preparar o terreno conceitual e técnico para o desenvolvimento prático, treinamento e otimização dos modelos na etapa subsequente do projeto (TCC 2), garantindo que as escolhas arquiteturais sejam informadas por evidências científicas e não por modismos tecnológicos.
    
    \item \textbf{Implementar, treinar, otimizar e comparar os modelos preditivos (Planejamento para TCC 2):} Desenvolver, treinar e otimizar sistematicamente os três algoritmos de previsão selecionados. Para o SARIMA, implementar seleção automática de hiperparâmetros (ordens p, d, q para componente não-sazonal e P, D, Q, s para componente sazonal) utilizando critérios de informação (AIC, BIC) e validação cruzada. Para o XGBoost, realizar otimização de hiperparâmetros (taxa de aprendizado, profundidade máxima das árvores, número de estimadores, regularização L1 e L2) através de busca em grade (\textit{Grid Search}) ou busca aleatória (\textit{Randomized Search}) combinada com validação cruzada. Para a LSTM, projetar arquitetura apropriada (número de camadas, número de unidades de memória por camada, camadas de dropout para prevenção de overfitting, camadas densas para regressão final) e otimizar hiperparâmetros (taxa de aprendizado, tamanho do batch, número de épocas, otimizador) utilizando técnicas de early stopping. Utilizar exclusivamente validação cruzada em janela deslizante (\textit{Time Series Cross-Validation} ou \textit{Walk-Forward Validation}) para garantir que o treinamento ocorra sempre utilizando dados do passado e a avaliação ocorra utilizando dados do futuro, respeitando rigorosamente a ordem temporal e evitando vazamento de dados (\textit{data leakage}) que invalidaria as estimativas de desempenho.
    
    \item \textbf{Avaliar rigorosamente o desempenho preditivo e a capacidade de generalização dos modelos (Planejamento para TCC 2):} Calcular múltiplas métricas robustas de erro e acurácia preditiva para todos os modelos desenvolvidos, incluindo RMSE (Raiz do Erro Quadrático Médio, que penaliza erros grandes), MAE (Erro Absoluto Médio, de fácil interpretação), MAPE (Erro Percentual Absoluto Médio, para análise relativa), e métricas específicas para séries temporais como a direção de acurácia (capacidade de prever se os casos vão aumentar ou diminuir). Realizar análise detalhada de resíduos (erros de previsão) para identificar padrões sistemáticos de subestimação ou superestimação, heterocedasticidade (variação do erro ao longo do tempo), e autocorrelação residual que indicaria modelos mal especificados. Com foco específico e crítico na capacidade de generalização e previsão do evento extremo de 2024 (que pode ser considerado um "teste de estresse" dos modelos), avaliar se os modelos treinados com dados anteriores conseguem capturar adequadamente a magnitude e o timing desse surto sem precedentes. Comparar criticamente os resultados entre os três modelos, identificando não apenas qual abordagem oferece melhor desempenho médio, mas também sob quais condições específicas cada modelo se destaca (ex: XGBoost pode ser melhor para prever picos, enquanto LSTM pode ser melhor para capturar tendências de longo prazo). Documentar todas as limitações identificadas e áreas para melhoria futura.
\end{enumerate}

Esses objetivos específicos foram projetados para serem cumulativos e progressivos, onde cada etapa fornece insumos essenciais para as etapas subsequentes, garantindo que o trabalho final seja fundamentado em bases sólidas de dados de qualidade, análises estatísticas rigorosas, e modelagem preditiva tecnicamente robusta e cientificamente validada.
