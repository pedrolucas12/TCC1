\chapter[Introdução]{Introdução}
\label{cap:introducao}

\section{Importância do Trabalho e Contextualização}

\subsection{Por que este trabalho é importante?}

A dengue representa um dos desafios de saúde pública global da atualidade, especialmente em regiões tropicais e subtropicais. A Organização Mundial da Saúde (OMS) estima que aproximadamente 3,9 bilhões de pessoas — equivalente a metade da população mundial — vivem em áreas com risco de infecção por dengue. A incidência global da doença aumentou vertiginosamente nas últimas quatro décadas, passando de cerca de meio milhão de casos notificados em 2000 para 5,2 milhões em 2019, representando um crescimento superior a 1000\% \cite{naish2014climate}.

No Brasil, o cenário epidemiológico caracteriza-se por uma hiperendemicidade, com a co-circulação simultânea dos quatro sorotipos virais (DENV-1, DENV-2, DENV-3 e DENV-4). O país enfrenta epidemias cíclicas e sazonais que impõem uma carga severa e recorrente ao Sistema Único de Saúde (SUS) \cite{mussumeci2020large, roster2022machine}. Dados oficiais do Ministério da Saúde revelam que, apenas no primeiro semestre de 2024, foram notificados 6 milhões de casos prováveis de dengue em todo o território nacional, superando os registros históricos anteriores e evidenciando a dimensão continental do problema sanitário \cite{sinan2025}.

O Distrito Federal (DF), objeto de estudo desta pesquisa, apresenta características epidemiológicas e climáticas relevantes para análise da dinâmica de transmissão da dengue. Localizado no bioma Cerrado, o DF apresenta características climáticas específicas que modulam sazonalmente a dinâmica de transmissão: um regime pluviométrico bem definido, alternando entre uma estação seca (aproximadamente de maio a setembro) e uma chuvosa (outubro a abril), criando condições cíclicas para a proliferação do vetor \textit{Aedes aegypti}.

Dados consolidados e processados para esta pesquisa, obtidos através da integração sistemática das bases do Sistema de Informação de Agravos de Notificação (SINAN) e do Instituto Nacional de Meteorologia (INMET) \cite{sinan2025, inmet2025}, cobrem o período de 2000 a 2024. Dentre os anos analisados, o ano de 2024 caracterizou-se por uma epidemia sem precedentes históricos recentes no Distrito Federal, totalizando \textbf{313.317 casos} notificados. Este volume representa um aumento de 360\% em relação ao ano anterior (64.897 casos em 2023) e supera a soma de toda a série histórica recente dos anos anteriores. A média semanal de casos passou de aproximadamente 1.247 casos por semana em 2022 para 5.927 casos por semana em 2024, evidenciando uma mudança de escala epidemiológica.

O pico dessa crise ocorreu na semana epidemiológica iniciada em 18 de fevereiro de 2024, quando o sistema de saúde registrou \textbf{25.714 casos em apenas sete dias} — uma magnitude que evidencia a incapacidade dos métodos tradicionais de vigilância epidemiológica em antecipar eventos extremos, e também o colapso subsequente dos serviços de saúde. Relatos da mídia e documentos oficiais da Secretaria de Saúde do DF descreveram unidades hospitalares operando além da capacidade, com pacientes aguardando atendimento em corredores, esgotamento de insumos básicos (soro fisiológico, medicamentos para controle de dor e febre), e comprometimento do atendimento a outras patologias urgentes, incluindo emergências cardíacas e neurológicas.

Este trabalho é importante porque busca transicionar o paradigma da vigilância epidemiológica de uma abordagem \textbf{reativa} ("o que aconteceu?") para uma abordagem \textbf{preditiva e proativa} ("o que vai acontecer e quando?"). Os métodos tradicionais de vigilância operam com base em notificações clínicas e laboratoriais, que são naturalmente defasadas em relação ao momento real da transmissão viral na comunidade. Existe um intervalo crítico de várias semanas entre a ocorrência de um evento ambiental (ex: chuva intensa) e a manifestação clínica dos primeiros casos, período durante o qual a transmissão já está ocorrendo silenciosamente.

\subsection{Do que se trata o assunto abordado?}

A dengue é uma arbovirose sistêmica de evolução benigna na maioria dos casos (aproximadamente 70-80\%), mas que pode apresentar manifestações graves e potencialmente letais. O agente etiológico é um vírus de RNA de fita simples e polaridade positiva (DENV), pertencente à família \textit{Flaviviridae} e ao gênero \textit{Flavivirus}, que compartilha características filogenéticas com outros vírus de importância médica, como os causadores da febre amarela, Zika e chikungunya.

O vírus da dengue possui quatro sorotipos antigenicamente distintos e geneticamente relacionados (DENV-1, DENV-2, DENV-3 e DENV-4), cada um com características epidemiológicas e patogênicas próprias. Estudos moleculares revelam que esses sorotipos divergiram de um ancestral comum há aproximadamente 1.000 anos, e sua diferenciação continua através de mutações acumuladas ao longo do tempo, gerando diferentes genótipos e linhagens dentro de cada sorotipo.

A transmissão ocorre primariamente pela picada de fêmeas infectadas de mosquitos do gênero \textit{Aedes}, sendo o \textit{Aedes aegypti} o vetor primário em áreas urbanas e o \textit{Aedes albopictus} um vetor secundário com potencial crescente de adaptação a ambientes periurbanos e rurais. O \textit{Aedes aegypti} é uma espécie altamente antropofílica, demonstrando preferência marcada por se alimentar de sangue humano, o que aumenta sua eficiência como vetor em áreas urbanas densamente povoadas. O mosquito se infecta ao picar um indivíduo virêmico (portador do vírus na corrente sanguínea) e, após um período de incubação extrínseco (PIE) que varia de 8 a 12 dias dependendo da temperatura ambiente, torna-se capaz de transmitir o vírus a novos hospedeiros humanos durante múltiplos repastos sanguíneos ao longo de sua vida adulta, que pode durar de 2 a 4 semanas em condições ideais.


\begin{figure}[htb]
	\centering
	\caption{\textit{Aedes aegypti}, principal vetor da dengue em áreas urbanas}
	\includegraphics[width=0.8\textwidth]{figuras/Mosquito da Dengue.jpg}
	\label{fig:aedes_aegypti}
	\fonte{Ministério da Saúde, 2022.}
\end{figure}

A Figura~\ref{fig:aedes_aegypti} apresenta o mosquito \textit{Aedes aegypti}, facilmente identificável por suas listras brancas características nas pernas e uma marca em forma de lira no tórax, características morfológicas que auxiliam na identificação visual durante ações de vigilância entomológica.

A infecção por um sorotipo confere imunidade permanente e específica apenas para aquele sorotipo particular, havendo a possibilidade de infecções subsequentes pelos demais sorotipos. Essa característica imunológica aumenta o risco de desenvolvimento de formas graves da doença em infecções secundárias. A teoria aceita, conhecida como "hipótese de potência dependente de anticorpos" (Antibody-Dependent Enhancement - ADE), postula que anticorpos gerados em uma primeira infecção podem, paradoxalmente, facilitar a entrada do vírus heterotípico nas células do sistema imune (monócitos e macrófagos) durante uma segunda infecção, aumentando a carga viral e desencadeando uma resposta imune exacerbada. Essa resposta imune hiperativa pode levar a complicações graves, como extravasamento plasmático maciço, hemorragias espontâneas e choque hipovolêmico, características da Dengue Grave (anteriormente denominada Dengue Hemorrágica), que pode evoluir para a síndrome do choque da dengue e óbito caso não seja tratada adequadamente e em tempo hábil.

\subsubsection{Gravidade da doença e limitações impostas aos pacientes}

Os pacientes com dengue enfrentam limitações físicas, funcionais e sociais durante o período da doença. A fase febril inicial, que dura tipicamente de 3 a 7 dias, caracteriza-se por sintomas súbitos e intensos que aparecem abruptamente após o período de incubação intrínseca (4 a 10 dias após a picada do mosquito infectado). A febre alta súbita ($38-40~^\circ\mathrm{C}$), muitas vezes bifásica (com dois picos de temperatura), é acompanhada de dores musculares intensas e generalizadas que os pacientes frequentemente descrevem como "dor nos ossos", cefaleia severa e persistente, e dor retroorbital (atrás dos olhos) que piora com movimentos oculares. Esses sintomas comprometem a capacidade de realizar atividades cotidianas básicas, como trabalhar, estudar ou mesmo realizar tarefas domésticas simples.

Em casos graves, que ocorrem em aproximadamente 5-10\% das infecções e são mais comuns em infecções secundárias, os pacientes podem apresentar manifestações clínicas críticas. A magnitude do problema pode ser observada nos dados nacionais de incidência e óbitos por dengue, ilustrados na Figura~\ref{fig:dengue_incidencia_obitos}, que evidenciam a severidade crescente das epidemias no Brasil.

\begin{figure}[htb]
	\centering
	\caption{Incidência e óbitos por dengue no Brasil (janeiro de 2025)}
	\includegraphics[width=0.6\textwidth]{figuras/Dengue Incidencia e Obitos Jan 2025.png}
	\label{fig:dengue_incidencia_obitos}
	\fonte{SINAN, IBGE, 2025.}
\end{figure}

As manifestações clínicas críticas incluem:

\begin{itemize}
    \item \textbf{Extravasamento de plasma:} O mecanismo fisiopatológico central da dengue grave é o aumento da permeabilidade vascular, resultando em extravasamento de plasma para os espaços extravasculares. Clinicamente, isso se manifesta como derrame pleural (acúmulo de líquido na cavidade torácica), ascite (acúmulo de líquido no abdome), derrame pericárdico e edema generalizado. Em casos severos, a perda de volume plasmático pode ultrapassar 20\% do volume sanguíneo total, levando a hipotensão arterial, taquicardia compensatória e, em última instância, choque hipovolêmico, exigindo hospitalização imediata em unidade de terapia intensiva, reposição volêmica agressiva com soluções cristaloides ou coloides, e monitoramento hemodinâmico contínuo.
    
    \item \textbf{Hemorragias:} A trombocitopenia (diminuição do número de plaquetas) e as alterações na função plaquetária, combinadas com distúrbios da coagulação, podem resultar em sangramentos espontâneos ou após procedimentos médicos (venopunção, injeções intramusculares). As manifestações hemorrágicas variam desde petéquias cutâneas e epistaxe (sangramento nasal) até hemorragia gastrointestinal, sangramento uterino anormal e, em casos extremos, hemorragia intracraniana, que pode ser fatal se não diagnosticada e tratada prontamente. Pacientes com sangramentos ativos podem necessitar de transfusão de plaquetas e plasma fresco congelado.
    
    \item \textbf{Comprometimento de órgãos:} A dengue pode causar disfunção orgânica múltipla. Hepatite por dengue, caracterizada por elevação de transaminases e icterícia, pode levar a insuficiência hepática aguda. A encefalite por dengue, resultante da invasão viral do sistema nervoso central ou de complicações da síndrome de choque, pode causar convulsões, alteração do nível de consciência e déficits neurológicos permanentes. Miocardite e insuficiência cardíaca aguda têm sido relatadas, especialmente em crianças. Insuficiência renal aguda pode ocorrer secundária ao choque ou diretamente por nefrotoxicidade viral, exigindo terapia dialítica temporária em casos severos.
\end{itemize}

Além das manifestações clínicas agudas, muitos pacientes desenvolvem sequelas de longo prazo que podem persistir por semanas ou meses após a resolução da fase aguda da infecção. A síndrome pós-dengue, ainda pouco estudada mas frequentemente relatada por pacientes e profissionais de saúde, caracteriza-se por fadiga crônica incapacitante, dores articulares e musculares persistentes, dificuldades de concentração e memória (popularmente conhecida como "neblina mental"), e depressão secundária. Estudos longitudinais sugerem que até 40\% dos pacientes que tiveram dengue grave podem apresentar sintomas persistentes após 3 meses da infecção aguda, impactando a qualidade de vida e a capacidade de retorno às atividades profissionais e acadêmicas.

\subsubsection{Perigos e impacto social}

O impacto da dengue transcende a esfera clínica individual, gerando um fardo multidimensional para a sociedade, a economia e o sistema de saúde. Epidemias de grande magnitude, como a observada no Distrito Federal em 2024, provocam uma cascata de efeitos deletérios que se propagam através de múltiplas camadas sociais e econômicas.

\paragraph{Ocupação e sobrecarga hospitalar}

A sobrecarga do Sistema Único de Saúde (SUS) é um dos impactos visíveis e imediatos. Durante picos epidêmicos, as unidades de atenção primária, pronto-atendimento e emergências hospitalares são inundadas com pacientes procurando atendimento, muitas vezes com sintomas leves ou moderados que poderiam ser manejados ambulatorialmente, mas que geram ansiedade e procura por serviços de saúde devido ao medo de complicações. Isso resulta em tempos de espera que podem ultrapassar 8-12 horas, comprometendo a qualidade do atendimento e aumentando o risco de erros médicos por sobrecarga dos profissionais. Leitos hospitalares, especialmente em unidades de terapia intensiva pediátrica e adulta, são rapidamente ocupados por pacientes com dengue grave, reduzindo a disponibilidade para outras emergências médicas, incluindo acidentes vasculares cerebrais, infartos do miocárdio e traumas. Durante a epidemia de 2024 no Distrito Federal, unidades de saúde relataram ocupação de leitos clínicos acima de 95\%, com pacientes sendo acomodados em corredores e espaços improvisados.

\paragraph{Mortalidade e impacto em vidas humanas}

Embora a maioria dos casos de dengue seja benigna, a mortalidade associada à doença representa um fardo humano imensurável. Óbitos por dengue ocorrem principalmente em casos graves que evoluem para síndrome do choque da dengue ou hemorragia grave, frequentemente associados a diagnóstico tardio ou acesso inadequado a cuidados de saúde. Durante a epidemia de 2024 no Distrito Federal, foram registrados dezenas de óbitos confirmados por dengue, muitos deles potencialmente evitáveis com diagnóstico precoce e tratamento adequado. Crianças, idosos e pessoas com comorbidades (diabetes, hipertensão arterial, doenças cardiovasculares) apresentam maior risco de evolução para formas graves e óbito. Além dos óbitos diretos, a dengue pode precipitar complicações em pacientes com condições preexistentes, contribuindo indiretamente para o aumento da mortalidade geral durante períodos epidêmicos.

\paragraph{Absenteísmo escolar e laboral}

O absenteísmo escolar e laboral durante epidemias atinge níveis críticos. Estudos econômicos estimam que, durante a epidemia de 2024 no DF, milhares de dias letivos foram perdidos por estudantes que estiveram doentes ou cuidando de familiares doentes. Escolas públicas e privadas reportaram ausências massivas de alunos e professores durante os picos da epidemia, comprometendo o calendário letivo e a qualidade do ensino. No ambiente de trabalho, a perda de produtividade é significativa: funcionários doentes ficam afastados por períodos que variam de 5 a 15 dias, dependendo da gravidade, enquanto outros podem precisar se ausentar para cuidar de filhos ou parentes doentes. Em setores essenciais, como educação, segurança pública e saúde, essas ausências podem comprometer o funcionamento básico dos serviços. O setor de serviços, incluindo comércio e turismo, também sofre impactos significativos durante epidemias, com redução no fluxo de clientes e turistas em áreas afetadas.

\paragraph{Custos econômicos diretos e indiretos}

Os custos econômicos diretos e indiretos são elevados. Estudos de análise de custo-efetividade estimam que o custo médio por caso de dengue no Brasil varia de R\$ 500 a R\$ 2.000, dependendo da complexidade do caso e da necessidade de hospitalização. Multiplicando esses valores pelos milhões de casos anuais, chega-se a um impacto econômico que pode ultrapassar R\$ 5 bilhões anualmente. Os custos diretos incluem gastos com consultas médicas, exames laboratoriais (sorologias, hemogramas, exames de função hepática e renal), medicamentos (analgésicos, antitérmicos, soluções de reidratação), internações hospitalares (incluindo custos de leitos de enfermaria e UTI), procedimentos diagnósticos e terapêuticos (transfusões de sangue e derivados, diálise em casos graves), e custos operacionais do sistema de saúde (sobrecarga de profissionais, horas extras, contratação temporária). Os custos indiretos, frequentemente subestimados, incluem perda de produtividade da força de trabalho, custos previdenciários por afastamentos (auxílio-doença, benefício previdenciário), impacto no turismo (áreas afetadas podem ter redução de visitantes e cancelamento de eventos), redução na arrecadação de impostos devido à queda na atividade econômica, e custos de oportunidade (recursos que poderiam ser investidos em outras áreas da saúde pública ou desenvolvimento social).

\paragraph{Desigualdades sociais e vulnerabilidade}

Populações vulneráveis, especialmente aquelas em áreas urbanas periféricas com infraestrutura sanitária precária, são desproporcionalmente afetadas, exacerbando desigualdades sociais existentes. Nessas comunidades, fatores como falta de abastecimento regular de água (levando ao armazenamento doméstico em recipientes que se tornam criadouros), coleta irregular de lixo (que acumula materiais que retêm água da chuva), e habitação precária (com telhados que permitem entrada de água) criam condições ideais para a proliferação do vetor. Além disso, o acesso limitado a serviços de saúde de qualidade, especialmente em horários de crise, significa que casos graves podem não receber atenção adequada em tempo hábil, aumentando o risco de complicações e óbitos evitáveis. Durante a epidemia de 2024 no Distrito Federal, dados mostraram que regiões administrativas com menor Índice de Desenvolvimento Humano (IDH) apresentaram taxas de incidência e letalidade por dengue superiores às áreas mais desenvolvidas, evidenciando a dimensão social e de equidade do problema.

\subsection{O que é Aprendizado de Máquina?}

Para profissionais de saúde pública, epidemiologistas e gestores que não possuem formação técnica em ciência da computação ou estatística avançada, é fundamental compreender de forma clara e didática o que são os métodos de aprendizado de máquina (\textit{Machine Learning}) e como eles podem ser aplicados para melhorar a vigilância epidemiológica da dengue. Esta subseção busca fornecer uma explicação acessível desses conceitos, utilizando analogias que facilitem a compreensão por parte de leitores com formação em áreas biológicas e médicas.

O aprendizado de máquina pode ser compreendido como uma abordagem computacional que permite que sistemas automatizados identifiquem padrões complexos em grandes volumes de dados históricos, aprendendo com exemplos passados para fazer previsões sobre eventos futuros. De forma análoga ao processo de aprendizado humano através da experiência, algoritmos de aprendizado de máquina são "treinados" utilizando dados históricos conhecidos (por exemplo, casos de dengue e variáveis climáticas dos últimos 25 anos no Distrito Federal), identificando relações e padrões que não seriam facilmente perceptíveis através de análise manual ou métodos estatísticos tradicionais. Após esse treinamento, o sistema é capaz de aplicar o conhecimento adquirido para fazer previsões sobre cenários novos e não observados anteriormente.

Para ilustrar esse conceito com uma analogia familiar ao contexto médico, imagine um médico experiente que, após anos de prática clínica atendendo pacientes com dengue, desenvolve uma capacidade intuitiva de reconhecer padrões sutis que indicam maior risco de complicações graves: combinações específicas de sintomas, histórico epidemiológico, características demográficas e fatores ambientais que, quando presentes simultaneamente, sugerem necessidade de monitoramento mais intensivo. O aprendizado de máquina opera de forma similar, mas de maneira sistemática e quantitativa: analisa milhares ou milhões de casos históricos, identifica quais combinações de variáveis (temperatura, precipitação, umidade, casos anteriores) estão associadas a surtos epidêmicos, e utiliza esse conhecimento para alertar sobre riscos futuros antes que os casos clínicos comecem a aparecer nos sistemas de notificação.

Uma característica fundamental que diferencia o aprendizado de máquina de métodos estatísticos tradicionais é sua capacidade de capturar relações não-lineares e interações complexas entre múltiplas variáveis simultaneamente. Enquanto análises estatísticas clássicas frequentemente assumem relações lineares (por exemplo, "um aumento de 1°C na temperatura resulta em um aumento proporcional de X casos de dengue"), o aprendizado de máquina pode identificar padrões mais sofisticados, como: "quando a temperatura está entre 24°C e 28°C E a precipitação acumulada nas últimas 4 semanas excede 150mm E a umidade relativa está acima de 70\%, então há alta probabilidade de um surto ocorrer nas próximas 6-8 semanas, mas esse padrão não se aplica se a temperatura anterior foi muito baixa". Essas relações complexas são particularmente relevantes para a dengue, uma vez que a dinâmica de transmissão envolve múltiplos fatores biológicos interconectados: o ciclo de vida do mosquito, o período de incubação viral, a disponibilidade de criadouros, e a suscetibilidade da população humana.

No contexto específico deste trabalho, duas abordagens principais de aprendizado de máquina serão exploradas e comparadas. O primeiro método, denominado SARIMA (um modelo estatístico clássico que incorpora componentes sazonais), funciona de forma similar a um modelo epidemiológico matemático tradicional, identificando padrões recorrentes ao longo do tempo (como sazonalidade anual) e utilizando esses padrões para projetar tendências futuras. O segundo método, XGBoost, é um algoritmo de aprendizado de máquina que pode ser comparado a um sistema de decisão em árvore múltipla: o algoritmo constrói uma série de "regras de decisão" hierárquicas (semelhantes a fluxogramas clínicos) que avaliam diferentes combinações de variáveis climáticas e epidemiológicas para determinar o número esperado de casos. Cada "árvore" aprende a corrigir os erros das anteriores, resultando em um sistema de previsão progressivamente mais preciso.

É importante destacar que o aprendizado de máquina não substitui o conhecimento especializado de profissionais de saúde ou epidemiologistas, mas sim amplifica e sistematiza a capacidade de análise, permitindo processar volumes de dados que seriam impraticáveis para análise manual. Da mesma forma que exames laboratoriais e de imagem complementam o diagnóstico clínico sem substituir o julgamento médico, modelos preditivos baseados em aprendizado de máquina fornecem informações quantitativas adicionais que podem fundamentar decisões de saúde pública, como alocação de recursos, intensificação de controle vetorial, e preparação de unidades de saúde para períodos de alta demanda. A transparência e interpretabilidade dos modelos desenvolvidos neste trabalho são aspectos prioritários, garantindo que os resultados possam ser compreendidos e validados por profissionais de saúde pública, e não funcionem como "caixas-pretas" inacessíveis.

\subsection{A dengue e a metodologia proposta}

Diante do cenário descrito, caracterizado por epidemias cíclicas e recorrentes que impõem custos humanos, sociais e econômicos elevados ao Distrito Federal e ao Brasil, torna-se imperativo desenvolver estratégias inovadoras de vigilância epidemiológica que permitam antecipar surtos antes de sua manifestação clínica completa. A metodologia proposta neste trabalho foi estruturada para responder a essa necessidade, estabelecendo uma abordagem sistemática e fundamentada em evidências para o desenvolvimento de modelos preditivos de surtos de dengue.

A metodologia proposta, detalhada no Capítulo~\ref{cap:metodologia}, foi desenhada considerando as características específicas da dinâmica de transmissão da dengue no Distrito Federal, utilizando dados históricos do SINAN (desde 2007) e do INMET (desde 2000), totalizando um período de análise de 25 anos. Esta extensão temporal permite capturar a variabilidade climática de longo prazo e os padrões epidemiológicos que caracterizam diferentes períodos, incluindo anos epidêmicos e não-epidêmicos.

A primeira fase da metodologia concentra-se na consolidação e expansão da base de dados, integrando sistematicamente dados epidemiológicos e climáticos através de um pipeline automatizado e reprodutível. Esta integração é fundamental, pois a literatura científica estabelece consistentemente que variáveis climáticas (precipitação, temperatura, umidade relativa) modulam a dinâmica de transmissão da dengue através de mecanismos biológicos bem compreendidos: a temperatura influencia o ciclo de vida do vetor e o período de incubação do vírus, a precipitação fornece criadouros para o desenvolvimento do mosquito, e a umidade afeta a sobrevivência e atividade dos mosquitos adultos. A disponibilidade de séries históricas extensas tanto para dados epidemiológicos quanto climáticos no Distrito Federal cria condições ideais para investigar essas relações e desenvolver modelos preditivos robustos.

A segunda fase da metodologia, de engenharia de atributos e análise exploratória, reconhece que os efeitos climáticos sobre a incidência de dengue não são imediatos, mas ocorrem com defasagens temporais (lags) que refletem o ciclo biológico do mosquito e os processos de incubação viral. A construção sistemática de atributos defasados (de 1 a 12 semanas) e de atributos derivados (médias móveis, anomalias climáticas, indicadores sazonais) permite que os modelos capturem relações temporais complexas entre variáveis ambientais e casos de dengue. A análise exploratória incluirá testes estatísticos formais de correlação e causalidade de Granger, que fornecerão evidências quantitativas sobre quais variáveis climáticas e quais defasagens são mais relevantes para prever surtos no contexto do Distrito Federal.

A terceira fase, de desenvolvimento de modelos preditivos, propõe a comparação sistemática de diferentes arquiteturas (modelos estatísticos clássicos, machine learning e deep learning) treinadas e avaliadas utilizando os mesmos dados históricos do Distrito Federal. Esta comparação permitirá identificar qual abordagem oferece melhor capacidade preditiva para diferentes cenários (previsão de curto, médio e longo prazo), considerando as particularidades da dinâmica de transmissão da dengue no bioma Cerrado. A validação temporal walk-forward garantirá que os modelos sejam avaliados em condições realistas, simulando o cenário de produção onde previsões são geradas utilizando apenas informações históricas.

Finalmente, a quarta fase da metodologia contempla a integração dos modelos em um sistema de alerta operacional, com desenvolvimento de API REST e dashboard interativo que permitirão aos gestores de saúde pública acessar previsões atualizadas semanalmente. Esta fase reconhece que modelos preditivos, por mais sofisticados que sejam, só produzem impacto real se seus resultados forem traduzidos em ações concretas de preparação e resposta do sistema de saúde.

A metodologia proposta está alinhada com os dados reais já disponíveis no repositório desta pesquisa, garantindo que todas as etapas sejam executáveis e reprodutíveis. Ao concentrar-se inicialmente no Distrito Federal e posteriormente expandir para todos os municípios brasileiros, a metodologia permite validação cuidadosa em escala local antes da expansão nacional, reduzindo riscos e garantindo qualidade do sistema final.

\section{Pergunta de Pesquisa}

Considerando o cenário epidemiológico no Distrito Federal, o potencial das variáveis ambientais como preditores antecipados, e a necessidade de transicionar de uma vigilância reativa para uma vigilância preditiva baseada em evidências quantitativas, este trabalho propõe-se a responder à seguinte questão central:

\begin{quote}
\textit{Como a integração sistemática de dados climáticos exógenos (precipitação, temperatura média, umidade relativa, pressão atmosférica) com a série temporal histórica de casos de dengue no Distrito Federal, utilizando janelas temporais defasadas (\textit{lags}) otimizadas estatisticamente através de análise de correlação cruzada e testes de causalidade de Granger, pode aprimorar a capacidade de modelos de Inteligência Artificial (SARIMA e XGBoost) em antecipar e dimensionar surtos epidêmicos severos, como o evento extremo observado em 2024, e de que forma essa capacidade preditiva pode fundamentar um sistema de alerta antecipado eficaz, preciso e robusto que os métodos tradicionais de vigilância epidemiológica baseados exclusivamente em notificações clínicas?}
\end{quote}

Esta pergunta de pesquisa sintetiza de forma abrangente o que está sendo proposto: não apenas desenvolver modelos preditivos como um exercício técnico, mas também investigar cientificamente, através de métodos estatísticos rigorosos, a contribuição real e quantificável das variáveis climáticas defasadas na capacidade de previsão. Além disso, a pergunta busca comparar sistematicamente diferentes arquiteturas de IA (um modelo estatístico clássico e um algoritmo de machine learning baseado em árvores) para identificar qual abordagem oferece melhor capacidade de previsão em diferentes cenários, expandindo o conhecimento do campo em modelagem preditiva aplicada à saúde pública e fornecendo evidências científicas sólidas para fundamentar decisões de política pública em vigilância epidemiológica.

\section{Objetivos}

Os objetivos deste trabalho constituem a resposta direta e operacional à pergunta de pesquisa formulada acima, decompondo a questão central em tarefas metodológicas específicas, mensuráveis e executáveis.

\subsection{Objetivo Geral}

Desenvolver, treinar, otimizar e avaliar comparativamente modelos preditivos baseados em Inteligência Artificial (SARIMA e XGBoost) para a previsão semanal de casos de dengue no Distrito Federal, utilizando uma base de dados unificada, tratada, validada e enriquecida com variáveis climáticas defasadas temporalmente do período de 2000 a 2024, visando fornecer uma ferramenta de alerta antecipado quantitativamente validada que demonstra capacidade preditiva superior aos métodos tradicionais de vigilância epidemiológica baseados exclusivamente em notificações clínicas tardias, contribuindo para a transição paradigmática da vigilância reativa para a vigilância preditiva e proativa na saúde pública brasileira.

\subsection{Objetivos Específicos}

Para alcançar o objetivo geral de forma estruturada e reprodutível, foram definidos os seguintes objetivos específicos, que representam os passos metodológicos sequenciais e interdependentes a serem seguidos:

\begin{enumerate}
    \item \textbf{Construir um pipeline automatizado, robusto e reprodutível de coleta e integração de dados:} Desenvolver scripts computacionais modulares e bem documentados em linguagem Python para a extração automatizada, limpeza sistemática, validação de qualidade e unificação estrutural de dados epidemiológicos do Sistema de Informação de Agravos de Notificação (SINAN), obtidos via API REST do InfoDengue, e dados meteorológicos históricos do Instituto Nacional de Meteorologia (INMET), referentes especificamente à Estação Meteorológica Automática de Brasília (A001) no período de 2000 a 2024. O pipeline deve garantir a padronização rigorosa de datas (formato ISO 8601), a agregação temporal adequada (conversão de dados diários/horários para granularidade semanal, utilizando soma para precipitação e média para variáveis contínuas), o tratamento de dados faltantes através de métodos apropriados (interpolação linear para variáveis climáticas, identificação e remoção de outliers impossíveis), e a compatibilidade estrutural entre as bases através de merge utilizando a data de início da semana epidemiológica como chave primária.
    
    \item \textbf{Realizar análise exploratória abrangente e análise estatística profunda de correlação e causalidade:} Quantificar sistematicamente a influência das variáveis climáticas (precipitação, temperatura média, umidade relativa, pressão atmosférica) sobre a incidência de dengue através do cálculo de correlações de Pearson (para relações lineares) e Spearman (para relações monotônicas não-lineares), implementar análise de correlação cruzada (Cross-Correlation Function - CCF) para identificar e quantificar os tempos de defasagem temporal (\textit{lags}) significativos e biologicamente relevantes para cada variável ambiental, explorando defasagens de 0 a 12 semanas para capturar o ciclo biológico completo do vetor e o período de incubação viral. Adicionalmente, aplicar testes estatísticos formais de causalidade de Granger para validar cientificamente a hipótese de que as variáveis climáticas antecedem temporalmente e melhoram a capacidade de previsão dos casos de dengue, comparando modelos autoregressivos com e sem a inclusão de valores defasados das covariáveis climáticas, utilizando nível de significância de 5\% e testando defasagens de 1 a 4 semanas. Gerar uma tabela consolidada de correlações e causalidades que sirva como referência para a engenharia de atributos na etapa de modelagem.
    
    \item \textbf{Realizar decomposição temporal avançada e análise detalhada de sazonalidade:} Aplicar decomposição STL (Seasonal and Trend decomposition using Loess) à série temporal de casos de dengue para isolar e analisar separadamente os componentes de tendência de longo prazo, sazonalidade recorrente (ciclos anuais) e resíduo (ruído aleatório e variação não-explicada), permitindo uma compreensão da estrutura temporal dos dados. Identificar padrões sazonais recorrentes (ex: picos consistentes em fevereiro-março), tendências de crescimento ou declínio de longo prazo, e eventos atípicos ou anomalias que se destacam do padrão esperado. Essa análise deve informar diretamente a escolha de hiperparâmetros dos modelos preditivos (ex: ordem de sazonalidade no SARIMA) e validar se os modelos conseguem capturar adequadamente esses componentes estruturais.
    
    \item \textbf{Fundamentar teoricamente e justificar a escolha das arquiteturas de IA selecionadas:} Realizar revisão bibliográfica sistemática e crítica sobre os dois algoritmos de previsão de séries temporais selecionados (SARIMA como modelo estatístico clássico de baseline, XGBoost como representante de algoritmos de machine learning baseados em ensemble de árvores de decisão), detalhando seus fundamentos matemáticos e computacionais, vantagens específicas para o problema de predição epidemiológica, limitações conhecidas, requisitos de implementação técnica, hiperparâmetros críticos que precisarão ser otimizados, estratégias de validação apropriadas, e evidências de aplicação bem-sucedida em problemas similares na literatura científica. Esta fundamentação teórica deve preparar o terreno conceitual e técnico para o desenvolvimento prático, treinamento e otimização dos modelos na etapa subsequente do projeto (TCC 2), garantindo que as escolhas arquiteturais sejam informadas por evidências científicas e não por modismos tecnológicos.
    
    \item \textbf{Implementar, treinar, otimizar e comparar os modelos preditivos (Planejamento para TCC 2):} Desenvolver, treinar e otimizar sistematicamente os dois algoritmos de previsão selecionados. Para o SARIMA, implementar seleção automática de hiperparâmetros (ordens p, d, q para componente não-sazonal e P, D, Q, s para componente sazonal) utilizando critérios de informação (AIC, BIC) e validação cruzada. Para o XGBoost, realizar otimização de hiperparâmetros (taxa de aprendizado, profundidade máxima das árvores, número de estimadores, regularização L1 e L2) através de busca em grade (\textit{Grid Search}) ou busca aleatória (\textit{Randomized Search}) combinada com validação cruzada. Utilizar exclusivamente validação cruzada em janela deslizante (\textit{Time Series Cross-Validation} ou \textit{Walk-Forward Validation}) para garantir que o treinamento ocorra sempre utilizando dados do passado e a avaliação ocorra utilizando dados do futuro, respeitando rigorosamente a ordem temporal e evitando vazamento de dados (\textit{data leakage}) que invalidaria as estimativas de desempenho.
    
    \item \textbf{Avaliar rigorosamente o desempenho preditivo e a capacidade de generalização dos modelos (Planejamento para TCC 2):} Calcular múltiplas métricas robustas de erro e acurácia preditiva para todos os modelos desenvolvidos, incluindo RMSE (Raiz do Erro Quadrático Médio, que penaliza erros grandes), MAE (Erro Absoluto Médio, de fácil interpretação), MAPE (Erro Percentual Absoluto Médio, para análise relativa), e métricas específicas para séries temporais como a direção de acurácia (capacidade de prever se os casos vão aumentar ou diminuir). Realizar análise detalhada de resíduos (erros de previsão) para identificar padrões sistemáticos de subestimação ou superestimação, heterocedasticidade (variação do erro ao longo do tempo), e autocorrelação residual que indicaria modelos mal especificados. Com foco específico e crítico na capacidade de generalização e previsão do evento extremo de 2024 (que pode ser considerado um "teste de estresse" dos modelos), avaliar se os modelos treinados com dados anteriores conseguem capturar adequadamente a magnitude e o timing desse surto sem precedentes. Comparar criticamente os resultados entre os dois modelos, identificando não apenas qual abordagem oferece melhor desempenho médio, mas também sob quais condições específicas cada modelo se destaca (ex: XGBoost pode ser melhor para prever picos, enquanto SARIMA pode ser melhor para capturar tendências de longo prazo e padrões sazonais). Documentar todas as limitações identificadas e áreas para melhoria futura.
\end{enumerate}

Esses objetivos específicos foram projetados para serem cumulativos e progressivos, onde cada etapa fornece insumos essenciais para as etapas subsequentes, garantindo que o trabalho final seja fundamentado em bases sólidas de dados de qualidade, análises estatísticas rigorosas, e modelagem preditiva tecnicamente robusta e cientificamente validada.
