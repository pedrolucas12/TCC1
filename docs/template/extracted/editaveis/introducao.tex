\chapter[Introdução]{Introdução}
\label{cap:introducao}

A dengue é uma doença viral sistêmica de evolução benigna na maioria dos casos, mas que pode apresentar manifestações graves, incluindo a síndrome do choque da dengue e óbito. O agente etiológico é um vírus de RNA de fita simples e polaridade positiva (DENV), pertencente à família \textit{Flaviviridae} e ao gênero \textit{Flavivirus}. Existem quatro sorotipos antigenicamente distintos (DENV-1, DENV-2, DENV-3 e DENV-4), e a infecção por um deles confere imunidade permanente apenas para aquele sorotipo específico, havendo a possibilidade de infecções subsequentes pelos demais, o que aumenta o risco de formas graves da doença.

A transmissão ocorre primordialmente pela picada de fêmeas infectadas de mosquitos do gênero \textit{Aedes}, sendo o \textit{Aedes aegypti} o vetor primário em áreas urbanas. A biologia do vetor é intrinsecamente ligada às condições ambientais: a temperatura influencia a velocidade de reprodução e o período de incubação viral, enquanto a chuva fornece os criadouros aquáticos essenciais para a fase larvária.

\section{Contextualização e Motivação}

Nas últimas décadas, a dengue consolidou-se como um dos mais críticos desafios de saúde pública global, especialmente em regiões tropicais e subtropicais. A urbanização desordenada, as mudanças climáticas e a mobilidade humana global têm facilitado a expansão da área de atuação do vetor.

No Brasil, o cenário epidemiológico é caracterizado pela hiperendemicidade, com ciclos epidêmicos recorrentes. O Distrito Federal (DF), foco deste estudo, tornou-se um cenário alarmante e emblemático dessa crise sanitária nos últimos anos. Localizado no bioma Cerrado, o DF apresenta um regime pluviométrico bem definido, alternando entre uma estação seca e uma chuvosa, o que modula sazonalmente a dinâmica da transmissão.

A motivação para este trabalho nasce da observação dos dados alarmantes recentes. Dados consolidados e processados para esta pesquisa revelam que, no triênio compreendido entre 2022 e 2024, o Distrito Federal acumulou um total de \textbf{463.560 casos notificados} de dengue.

O ano de 2024, em particular, representou um ponto de inflexão histórica. Diferente dos anos anteriores, que seguiram padrões sazonais esperados (2022 com 85.346 casos e 2023 com 64.897 casos), 2024 registrou uma epidemia explosiva e sem precedentes, totalizando \textbf{313.317 casos} notificados. Esse volume representa um aumento superior a 360\% em relação ao ano anterior e supera a soma de toda a série histórica recente. O pico dessa crise ocorreu na semana epidemiológica iniciada em 18 de fevereiro de 2024, quando o sistema de saúde registrou mais de 25 mil casos em apenas sete dias.

Esse cenário de colapso, com superlotação de hospitais e esgotamento de recursos, evidencia a incapacidade dos métodos tradicionais de vigilância em prever, com a antecedência necessária, a magnitude de tais eventos extremos. A reatividade do sistema de saúde, que muitas vezes atua apenas após a detecção do aumento de casos, resulta em uma resposta tardia e menos eficaz.

\section{Relevância da Tecnologia}

A complexidade da dinâmica de transmissão da dengue reside na sua natureza multifatorial e não-linear. Fatores climáticos como precipitação, temperatura, umidade relativa do ar e pressão atmosférica interagem de forma complexa para criar as condições ideais para a proliferação do mosquito. Além disso, existe um fenômeno crucial conhecido como "defasagem temporal" ou \textit{lag}: a chuva que ocorre hoje não resulta em mosquitos adultos imediatamente, mas sim semanas depois, após o ciclo de eclosão dos ovos e desenvolvimento das larvas.

Os métodos estatísticos clássicos muitas vezes falham em capturar essas nuances não-lineares e as interações complexas entre múltiplas variáveis exógenas. É nesse contexto que a tecnologia de Inteligência Artificial (IA) se torna uma ferramenta indispensável.

Algoritmos de Aprendizado de Máquina (\textit{Machine Learning}), como o \textit{Gradient Boosting} (XGBoost), e de Aprendizado Profundo (\textit{Deep Learning}), como as Redes Neurais Recorrentes (LSTM - \textit{Long Short-Term Memory}), possuem a capacidade intrínseca de modelar dependências temporais de longo prazo e identificar padrões sutis em grandes volumes de dados (\textit{Big Data}). A aplicação dessas tecnologias permite transformar dados brutos climáticos e epidemiológicos em inteligência preditiva acionável, oferecendo aos gestores públicos uma janela de oportunidade ("horizonte de previsão") para atuar preventivamente antes que o surto atinja seu pico.

\section{Pergunta de Pesquisa}

Considerando a gravidade do cenário epidemiológico no Distrito Federal e o potencial inexplorado das variáveis ambientais como preditores antecipados, este trabalho propõe-se a responder à seguinte questão central:

\begin{quote}
\textit{Como a integração de dados climáticos exógenos (precipitação, temperatura, umidade, pressão) com a série temporal histórica de casos de dengue no Distrito Federal, utilizando janelas temporais defasadas (\textit{lags}) otimizadas, pode aprimorar a capacidade de modelos de Inteligência Artificial em antecipar e dimensionar surtos epidêmicos severos, como o evento extremo observado em 2024?}
\end{quote}

\section{Objetivos}

\subsection{Objetivo Geral}

Desenvolver, treinar e avaliar modelos preditivos baseados em Inteligência Artificial (SARIMA, XGBoost e LSTM) para a previsão semanal de casos de dengue no Distrito Federal, utilizando uma base de dados unificada, tratada e enriquecida com variáveis climáticas do período de 2022 a 2024, visando fornecer uma ferramenta de alerta antecipado para a saúde pública.

\subsection{Objetivos Específicos}

Para alcançar o objetivo geral, foram definidos os seguintes objetivos específicos, que compõem a metodologia deste trabalho:

\begin{enumerate}
    \item \textbf{Coleta e Integração de Dados:} Construir um \textit{pipeline} computacional automatizado para a extração, limpeza e unificação de dados epidemiológicos (do Sistema de Informação de Agravos de Notificação - SINAN) e dados meteorológicos (do Instituto Nacional de Meteorologia - INMET) referentes ao Distrito Federal no período de 2022 a 2024.
    
    \item \textbf{Análise Exploratória e de Correlação:} Realizar uma análise estatística profunda para quantificar a influência das variáveis climáticas sobre a incidência de dengue, identificando através de correlação cruzada (\textit{Cross-Correlation}) e testes de causalidade de Granger quais são os tempos de defasagem (\textit{lags}) mais significativos para cada variável ambiental.
    
    \item \textbf{Desenvolvimento de Modelos (Planejamento TCC 2):} Implementar e otimizar algoritmos de previsão de séries temporais, comparando uma abordagem estatística clássica (SARIMA) com abordagens de Machine Learning baseadas em árvores (XGBoost) e Redes Neurais Profundas (LSTM).
    
    \item \textbf{Avaliação de Desempenho (Planejamento TCC 2):} Validar os modelos utilizando a técnica de validação cruzada em janela deslizante (\textit{Time Series Cross-Validation}) e métricas robustas de erro (RMSE, MAE), com foco específico na capacidade dos modelos de generalizar e prever o pico epidêmico extremo de 2024.
\end{enumerate}
