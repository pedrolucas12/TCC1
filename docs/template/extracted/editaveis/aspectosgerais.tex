\chapter{Revisão Bibliográfica e Fundamentação Teórica}
\label{cap:revisao}

Este capítulo apresenta a fundamentação teórica necessária para a compreensão do problema abordado, discutindo a epidemiologia da dengue, o impacto socioeconômico da doença e o estado da arte das tecnologias de previsão baseadas em Inteligência Artificial. Além disso, detalha-se a estrutura das bases de dados reais utilizadas neste estudo.

\section{A Dengue: Epidemiologia e Dinâmica de Transmissão}

A dengue é, atualmente, a arbovirose mais importante do mundo em termos de morbidade e mortalidade humana. O vírus (DENV) possui quatro sorotipos distintos (DENV-1, DENV-2, DENV-3 e DENV-4). A co-circulação desses sorotipos em uma mesma região é um fator de risco crítico, pois a infecção prévia por um sorotipo pode, em alguns casos, exacerbar a resposta imune em uma infecção secundária por um sorotipo diferente, levando a quadros de Dengue Grave (anteriormente denominada Dengue Hemorrágica).

A dinâmica de transmissão da doença é complexa e fortemente influenciada por fatores ambientais e sociais. O vetor \textit{Aedes aegypti} é um mosquito altamente adaptado ao ambiente urbano e peridomiciliar. Seu ciclo de vida compreende quatro fases: ovo, larva, pupa e adulto. As fases imaturas ocorrem necessariamente na água, enquanto a fase adulta é aérea.
\begin{itemize}
    \item \textbf{Influência da Chuva:} A precipitação é fundamental para a criação e manutenção de criadouros larvários (pneus, vasos, caixas d'água destampadas), aumentando a densidade vetorial.
    \item \textbf{Influência da Temperatura:} Temperaturas mais elevadas aceleram o metabolismo do mosquito, encurtando o tempo de desenvolvimento larvário e o período de incubação extrínseco do vírus dentro do vetor, o que aumenta a frequência de repasto sanguíneo e a taxa de transmissão.
\end{itemize}

\subsection{Impacto Social e Econômico}

O impacto da dengue transcende a esfera clínica individual, gerando um fardo pesado para a sociedade e para a economia. Epidemias de grande magnitude, como a observada no Distrito Federal em 2024, provocam:
\begin{itemize}
    \item \textbf{Sobrecarga do Sistema de Saúde:} A procura massiva por atendimento ambulatorial e internação esgota rapidamente os recursos hospitalares, leitos e insumos, prejudicando o atendimento a outras patologias.
    \item \textbf{Absenteísmo e Perda de Produtividade:} A doença afeta predominantemente a população economicamente ativa, gerando dias perdidos de trabalho e estudo, com impacto direto no PIB local.
    \item \textbf{Custos Diretos e Indiretos:} Os custos envolvem desde o tratamento dos pacientes e campanhas de controle vetorial até os custos previdenciários e a perda de qualidade de vida da população.
\end{itemize}

\section{Estado da Arte em Previsão Epidemiológica}

A capacidade de antecipar surtos de doenças infecciosas tem sido uma prioridade de pesquisa global. A evolução das técnicas de previsão pode ser dividida em abordagens clássicas e abordagens modernas baseadas em dados (\textit{Data-Driven}).

\subsection{Modelagem Estatística Clássica (SARIMA)}
O modelo SARIMA (\textit{Seasonal AutoRegressive Integrated Moving Average}) é uma extensão do modelo ARIMA que suporta explicitamente a sazonalidade univariada. Ele é amplamente utilizado como \textit{baseline} em estudos epidemiológicos devido à sua robustez teórica e capacidade de modelar tendências lineares e ciclos sazonais.
No entanto, sua principal limitação é a dificuldade em incorporar múltiplas variáveis exógenas (covariáveis climáticas) de maneira não-linear. O SARIMA assume que as relações passadas se repetem linearmente no futuro, o que pode ser insuficiente para prever surtos explosivos causados por anomalias climáticas complexas.

\subsection{Machine Learning e Algoritmos de Boosting (XGBoost)}
O \textit{Extreme Gradient Boosting} (XGBoost) representa o estado da arte em algoritmos baseados em árvores de decisão. Diferente dos modelos lineares, o XGBoost constrói um conjunto (\textit{ensemble}) de modelos fracos (árvores de decisão) de forma sequencial, onde cada novo modelo tenta corrigir os erros dos anteriores.
\textbf{Vantagens para o problema:}
\begin{itemize}
    \item Capacidade de capturar interações não-lineares complexas entre variáveis (ex: a chuva só aumenta a dengue se a temperatura estiver acima de um certo limiar).
    \item Robustez a dados faltantes (\textit{missing values}) e \textit{outliers}.
    \item Importância de Atributos (\textit{Feature Importance}): Permite identificar quais variáveis climáticas são mais relevantes para a previsão.
\end{itemize}

\subsection{Deep Learning e Redes Neurais Recorrentes (LSTM)}
As redes LSTM (\textit{Long Short-Term Memory}) são uma arquitetura especial de Redes Neurais Recorrentes (RNN) projetadas para superar o problema de "esquecimento" de dependências longas. Em séries temporais epidemiológicas, eventos ocorridos há vários meses (ex: um verão muito chuvoso) podem influenciar o tamanho da população de mosquitos meses depois. As células de memória da LSTM conseguem reter essa informação relevante por longos períodos, tornando-as ideais para modelar os efeitos defasados (\textit{lags}) do clima sobre a doença.

\section{Caracterização das Bases de Dados Reais}

Para garantir a validade ecológica e a aplicabilidade prática deste estudo, foram utilizados exclusivamente dados reais e oficiais referentes ao Distrito Federal.

\subsection{Dados Epidemiológicos (SINAN/InfoDengue)}
Os dados de notificação de dengue foram obtidos através da API do projeto InfoDengue, que cura e padroniza os dados do Sistema de Informação de Agravos de Notificação (SINAN) do Ministério da Saúde.
\begin{itemize}
    \item \textbf{Abrangência Temporal:} Semanas epidemiológicas de 2022 a 2024 (156 semanas contínuas).
    \item \textbf{Abrangência Espacial:} Município de Brasília (Geocódigo IBGE: 5300108).
    \item \textbf{Natureza do Dado:} Casos prováveis (soma de casos confirmados e suspeitos), refletindo a carga real sobre o sistema de saúde.
\end{itemize}

\subsection{Dados Meteorológicos (INMET)}
As variáveis climáticas foram extraídas do Banco de Dados Meteorológicos para Ensino e Pesquisa (BDMEP) do Instituto Nacional de Meteorologia (INMET).
\begin{itemize}
    \item \textbf{Fonte:} Estação Meteorológica Automática de Brasília (A001).
    \item \textbf{Variáveis Processadas:}
    \begin{enumerate}
        \item \textbf{Precipitação (Chuva):} Acumulado semanal (mm). Variável crítica para formação de criadouros.
        \item \textbf{Temperatura Média, Mínima e Máxima:} Médias semanais (°C). Variável crítica para a velocidade do ciclo viral.
        \item \textbf{Umidade Relativa do Ar:} Média semanal (%). Variável crítica para a sobrevivência do mosquito adulto (em ambientes secos, o mosquito desidrata e morre mais rápido).
        \item \textbf{Pressão Atmosférica:} Média semanal.
    \end{enumerate}
\end{itemize}
