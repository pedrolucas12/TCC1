\chapter{Cronograma de Execução}
\label{cap:cronograma}

Este capítulo apresenta o planejamento temporal detalhado das atividades do TCC 1 e do TCC 2, organizadas por etapas e com estimativa de duração em semanas.

\section{Atividades do TCC 1}

O TCC 1 foi dedicado à construção da base de dados e à análise exploratória preliminar. As atividades já executadas e suas durações estimadas foram:

\begin{enumerate}
    \item \textbf{Revisão Bibliográfica e Fundamentação Teórica} (4 semanas)
    \begin{itemize}
        \item Pesquisa sobre epidemiologia da dengue e dinâmica de transmissão.
        \item Revisão do estado da arte em modelos preditivos para saúde pública.
        \item Estudo das tecnologias de Machine Learning e Deep Learning aplicadas a séries temporais epidemiológicas.
    \end{itemize}
    
    \item \textbf{Identificação e Caracterização das Bases de Dados} (2 semanas)
    \begin{itemize}
        \item Mapeamento das fontes de dados oficiais (SINAN e INMET).
        \item Análise da estrutura, formato e qualidade dos dados brutos.
        \item Definição da estratégia de coleta e padronização.
    \end{itemize}
    
    \item \textbf{Desenvolvimento do Pipeline de Engenharia de Dados} (3 semanas)
    \begin{itemize}
        \item Programação de scripts para download automatizado (API InfoDengue e arquivos INMET).
        \item Implementação de rotinas de limpeza e tratamento de dados.
        \item Criação do módulo de agregação temporal (diário para semanal).
        \item Desenvolvimento do algoritmo de unificação (merge) das bases.
    \end{itemize}
    
    \item \textbf{Análise Estatística Exploratória} (3 semanas)
    \begin{itemize}
        \item Geração de estatísticas descritivas e visualizações exploratórias.
        \item Cálculo de correlações de Pearson e Spearman.
        \item Implementação da análise de correlação cruzada (lags de 0 a 8 semanas).
        \item Execução dos testes de causalidade de Granger.
    \end{itemize}
    
    \item \textbf{Redação e Documentação} (2 semanas)
    \begin{itemize}
        \item Redação dos capítulos de Introdução, Revisão Bibliográfica e Metodologia.
        \item Incorporação dos resultados preliminares e figuras geradas.
        \item Revisão textual e formatação conforme normas ABNT.
    \end{itemize}
\end{enumerate}

\textbf{Total estimado do TCC 1: 14 semanas}

\section{Planejamento para o TCC 2}

O TCC 2 focará na construção, treinamento e avaliação dos modelos preditivos. O cronograma previsto é:

\begin{enumerate}
    \item \textbf{Feature Engineering e Preparação dos Dados para Modelagem} (2 semanas)
    \begin{itemize}
        \item Criação explícita das variáveis de lag (chuva_lag_1, chuva_lag_2, etc.).
        \item Normalização e padronização das variáveis numéricas.
        \item Divisão dos dados em conjuntos de treino, validação e teste (respeitando ordem temporal).
    \end{itemize}
    
    \item \textbf{Implementação do Modelo Baseline (SARIMA)} (2 semanas)
    \begin{itemize}
        \item Seleção automática de hiperparâmetros (p, d, q)(P, D, Q).
        \item Treinamento do modelo e geração de previsões.
        \item Avaliação inicial do desempenho.
    \end{itemize}
    
    \item \textbf{Implementação e Otimização do XGBoost} (3 semanas)
    \begin{itemize}
        \item Configuração da estrutura de dados para o algoritmo.
        \item Otimização de hiperparâmetros via Grid Search ou Randomized Search.
        \item Treinamento e validação com Time Series Cross-Validation.
        \item Análise da importância das features (Feature Importance).
    \end{itemize}
    
    \item \textbf{Implementação e Otimização da Rede Neural LSTM} (4 semanas)
    \begin{itemize}
        \item Projeto da arquitetura (número de camadas, unidades de memória, dropout).
        \item Preparação dos dados em formato de sequências temporais.
        \item Treinamento com early stopping e monitoramento de overfitting.
        \item Ajuste fino de hiperparâmetros (learning rate, batch size, epochs).
    \end{itemize}
    
    \item \textbf{Comparação de Modelos e Análise de Resultados} (2 semanas)
    \begin{itemize}
        \item Cálculo das métricas de desempenho (RMSE, MAE) para todos os modelos.
        \item Análise dos erros de previsão (resíduos).
        \item Avaliação da capacidade de generalização (especialmente para o evento extremo de 2024).
        \item Comparação crítica dos resultados e discussão dos trade-offs.
    \end{itemize}
    
    \item \textbf{Redação Final e Apresentação} (3 semanas)
    \begin{itemize}
        \item Redação do capítulo de Resultados Finais e Discussão.
        \item Elaboração das Conclusões e Trabalhos Futuros.
        \item Preparação da apresentação para a banca examinadora.
        \item Revisão final e formatação do documento completo.
    \end{itemize}
\end{enumerate}

\textbf{Total estimado do TCC 2: 16 semanas}

\section{Visão Geral do Projeto}

O projeto completo (TCC 1 + TCC 2) está previsto para ser concluído em aproximadamente \textbf{30 semanas} (cerca de 7 a 8 meses), considerando dedicação parcial compatível com a carga horária acadêmica de um estudante de graduação.
