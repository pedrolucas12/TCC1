\chapter{Resultados Preliminares}
\label{cap:resultados}

Este capítulo apresenta os resultados da análise exploratória e estatística dos dados unificados de dengue e variáveis climáticas para o Distrito Federal no período de 2022 a 2024. A base de dados consolidada compreende 156 semanas epidemiológicas contínuas, sem lacunas temporais significativas, totalizando 463,560 casos notificados no triênio.

\section{Caracterização Descritiva da Série Temporal}

A análise descritiva revela uma disparidade impressionante entre os anos estudados. Enquanto 2022 e 2023 apresentaram perfis epidemiológicos relativamente controlados, com médias semanais de 1,641 e 1,224 casos, respectivamente, o ano de 2024 caracterizou-se por uma epidemia de magnitude sem precedentes, com média semanal de 6,143 casos.

O comportamento sazonal é claramente observável. Os meses de maior incidência histórica concentram-se no primeiro trimestre do ano, com picos típicos em fevereiro, período que coincide com o auge da estação chuvosa no Cerrado. Esse padrão é consistente com a biologia do vetor, que requer água estagnada para completar seu ciclo de desenvolvimento.

\section{Evolução Temporal e Sazonalidade}

A Figura \ref{fig:temporal} apresenta a série temporal completa de casos de dengue sobreposta à precipitação semanal acumulada. Observa-se uma correspondência visual entre períodos de maior volume pluviométrico e subsequentes picos de casos, porém com um deslocamento temporal (	extit{lag}).

\begin{figure}[H]
    \centering
    \includegraphics[width=0.95\textwidth]{figuras/evolucao_temporal_combinada.png}
    \caption{Evolução temporal dos casos de dengue e precipitação no Distrito Federal (2022-2024). O eixo esquerdo representa os casos notificados e o eixo direito a precipitação em milímetros.}
    \label{fig:temporal}
\end{figure}

O evento extremo de 2024 é evidente: a semana epidemiológica iniciada em 18/02/2024 registrou o pico histórico de 25,714 casos em apenas sete dias. Nesse momento crítico, as condições climáticas eram favoráveis à proliferação do vetor: umidade relativa do ar de 86.3\%, temperatura média de 21.6°C e precipitação acumulada de 55.2 mm.

\section{Decomposição STL e Componentes Temporais}

A decomposição STL (Seasonal and Trend decomposition using Loess) permite isolar três componentes fundamentais da série temporal: a tendência de longo prazo, a sazonalidade recorrente e o resíduo (ruído aleatório).

\begin{figure}[H]
    \centering
    \includegraphics[width=0.9\textwidth]{figuras/decomposicao_stl_casos.png}
    \caption{Decomposição STL da série temporal de casos de dengue, revelando a componente de tendência (aumento geral), sazonalidade (ciclos anuais) e resíduos.}
    \label{fig:stl}
\end{figure}

A análise da componente sazonal confirma que o padrão cíclico anual é robusto e previsível, com picos recorrentes concentrados entre as semanas epidemiológicas 6 e 14 (fevereiro a abril). A componente de tendência revela um crescimento de fundo, possivelmente relacionado à expansão urbana e ao aumento da população suscetível. Os resíduos mostram variabilidade não explicada pelos componentes sazonal e tendência, justificando a necessidade de incorporar variáveis climáticas externas na modelagem preditiva.

\section{Análise de Correlação}

\subsection{Correlação Linear (Pearson e Spearman)}

A análise de correlação linear foi conduzida para quantificar a força da associação entre as variáveis climáticas e o número de casos semanais. A Tabela \ref{tab:correlacao} apresenta os coeficientes de Pearson e Spearman calculados.

A variável climática que apresentou a maior correlação linear imediata com os casos foi a \textbf{Umidade}, com um coeficiente de Pearson de 0.3731. O coeficiente de Spearman, que mede associações monotônicas (não necessariamente lineares), foi ainda mais elevado para essa variável, sugerindo uma relação robusta e não-linear subjacente.

\begin{figure}[H]
    \centering
    \includegraphics[width=0.8\textwidth]{figuras/heatmap_correlacao_oficial.png}
    \caption{Matriz de correlação de Pearson entre todas as variáveis do estudo. Valores próximos de +1 indicam correlação positiva forte; valores próximos de -1, correlação negativa forte; valores próximos de 0, ausência de correlação linear.}
    \label{fig:heatmap}
\end{figure}

\subsection{Correlação Cruzada e Defasagem Temporal}

A análise de correlação cruzada foi fundamental para desvendar o fenômeno de defasagem temporal. Calculou-se a correlação entre a série de casos ($Y_t$) e as séries climáticas defasadas de $k$ semanas ($X_{t-k}$), para $k = 0, 1, ..., 8$.

\begin{figure}[H]
    \centering
    \includegraphics[width=0.8\textwidth]{figuras/analise_lags_crosscorr.png}
    \caption{Função de Correlação Cruzada: Correlação entre casos de dengue e precipitação defasada de 0 a 8 semanas. O aumento gradual da correlação com o aumento do lag sugere que a chuva leva tempo para manifestar seu efeito biológico sobre a população de mosquitos.}
    \label{fig:lags}
\end{figure}

Os resultados são reveladores: enquanto a correlação imediata (lag 0) entre chuva e casos é relativamente baixa ($r = 0.09$), a correlação aumenta progressivamente ao longo do tempo, atingindo seu pico aproximadamente 7 a 8 semanas após o evento chuvoso ($r \approx 0.27$). Esse padrão é biologicamente consistente com o ciclo de vida do mosquito: a chuva cria os criadouros, os ovos eclodem, as larvas se desenvolvem, e os mosquitos adultos emergem semanas depois, iniciando o ciclo de transmissão.

\section{Teste de Causalidade de Granger}

Para validar estatisticamente a precedência temporal e a causalidade no sentido de Granger, foram realizados testes formais de hipótese. Os resultados indicam que a \textbf{precipitação} apresenta evidência estatística de causalidade de Granger nos lags 1 e 2 semanas (valor-p < 0.05), confirmando que as chuvas antecedem e melhoram a previsão dos casos futuros.

Por outro lado, a umidade relativa, embora apresente a maior correlação linear imediata, não demonstrou causalidade de Granger estatisticamente significativa nos lags testados (1 a 4 semanas), sugerindo que sua relação com os casos pode ser mais indireta ou mediada por outras variáveis não observadas.

\section{Implicações para a Modelagem}

Os achados desta análise preliminar fornecem direcionamentos claros para a etapa de modelagem preditiva (TCC 2):

\begin{enumerate}
    \item \textbf{Variáveis Clave:} A precipitação e a umidade devem ser incorporadas aos modelos, mas com tratamentos diferentes: a chuva com defasagens de 1-2 semanas (devido ao Granger), e a umidade de forma mais imediata ou como variável de contexto.
    
    \item \textbf{Engenharia de Atributos:} Será necessário criar variáveis de lag explícitas (ex: `chuva_lag_1`, `chuva_lag_2`) para alimentar os algoritmos de Machine Learning.
    
    \item \textbf{Sazonalidade:} A decomposição STL confirma que os modelos devem ser capazes de capturar a sazonalidade anual, seja através de componentes sazonais explícitos (SARIMA) ou através de atributos temporais cíclicos (ML).
\end{enumerate}
